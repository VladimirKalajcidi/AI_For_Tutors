```latex
\documentclass{article}
\usepackage{enumitem}

\begin{document}

\title{Учебный план по Математике для Лёхи}
\author{Преподаватель}
\date{}

\maketitle

\section*{Уровень: Богатые родители}

\section{Цели обучения}
\begin{enumerate}
    \item Развитие математических навыков и умений ученика.
    \item Построение логического мышления и умения решать задачи.
    \item Подготовка к дальнейшему изучению математики на более продвинутом уровне.
\end{enumerate}

\section{План занятий}

\subsection{Тема 1: Арифметика}
\begin{itemize}
    \item \textbf{Цель:} Улучшить навыки в выполнении арифметических операций.
    \item Сложение, вычитание, умножение и деление целых чисел.
    \item Решение задач на применение арифметических действий.
\end{itemize}

\subsection{Тема 2: Геометрия}
\begin{itemize}
    \item \textbf{Цель:} Ознакомить с основными понятиями геометрии.
    \item Изучение геометрических фигур: круг, треугольник, прямоугольник.
    \item Вычисление площадей и периметров различных фигур.
\end{itemize}

\subsection{Тема 3: Уравнения и неравенства}
\begin{itemize}
    \item \textbf{Цель:} Научить решать уравнения и неравенства.
    \item Решение линейных уравнений и неравенств.
    \item Работа с системами уравнений.
\end{itemize}

\subsection{Тема 4: Функции}
\begin{itemize}
    \item \textbf{Цель:} Понять понятие функции и методы ее изучения.
    \item Определение функции, ее график.
    \item Решение задач, связанных с функциями.
\end{itemize}

\subsection{Тема 5: Вероятность и статистика}
\begin{itemize}
    \item \textbf{Цель:} Познакомить с основами вероятности и статистики.
    \item Изучение понятий вероятности и статистических характеристик.
    \item Решение задач на вероятность и статистику.
\end{itemize}

\section{Домашние задания}
\begin{enumerate}
    \item Решение практических задач по каждой теме.
    \item Самостоятельное изучение теории и дополнительных материалов.
\end{enumerate}

\end{document}
```