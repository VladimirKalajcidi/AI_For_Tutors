Данный учебный план представлен в простом LaTeX-стиле, без использования сложных макроопределений и пакетов для оформления. 

```latex
\documentclass[a4paper,12pt]{article}
\begin{document}

\title{Учебный план по физике для ученика Максим}
\author{Преподаватель}
\date{Дата}

\maketitle

\section*{Введение}
Уровень ученика: Хороший парень, старательный. Мы с ним занимаемся уже несколько месяцев и сейчас начнём проходить термодинамику.

\section{Темодинамика}
\subsection{Основные понятия термодинамики}
Цель: знакомство с основными понятиями и законами термодинамики.

\subsection{Первый закон термодинамики}
Цель: понимание и применение первого закона термодинамики.

\subsection{Второй закон термодинамики}
Цель: понимание и применение второго закона термодинамики.

\subsection{Термодинамические процессы}
Цель: знакомство с различными типами термодинамических процессов.

\subsection{Цикл Карно}
Цель: изучение и понимание цикла Карно.

\section{Заключение}
Обзор пройденной темы, повторение ключевых моментов, подготовка к следующему разделу физики.

\end{document}
```
Обратите внимание, что LaTeX требует компиляции для преобразования в читаемый формат, такой как PDF. В учебных планах каждая тема обычно включает в себя главные пункты, которые необходимо обсудить и усвоить, а также какие-то задачи для практического применения теоретических знаний.