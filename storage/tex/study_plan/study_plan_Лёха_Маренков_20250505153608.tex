Конечно, вот учебный план по предмету Математика для ученика Лёхи, уровень: Богатые родители:

\documentclass{article}
\usepackage{enumitem}

\begin{document}

\title{Учебный план по Математике}
\author{Преподаватель: Имя Фамилия}
\date{}

\maketitle

\section*{Ученик: Лёха}
\section*{Уровень: Богатые родители}

\begin{enumerate}[label=\textbf{\arabic*.}]

\item \textbf{Тема:} Основы арифметики
    \begin{itemize}
        \item \textbf{Цель:} Освоить основные арифметические операции (сложение, вычитание, умножение, деление).
        \item \textbf{Порядок занятий:}
            \begin{enumerate}
                \item Введение в арифметику
                \item Сложение и вычитание чисел
                \item Умножение и деление чисел
                \item Практические задания
            \end{enumerate}
    \end{itemize}

\item \textbf{Тема:} Геометрия
    \begin{itemize}
        \item \textbf{Цель:} Изучение основных геометрических фигур и их свойств.
        \item \textbf{Порядок занятий:}
            \begin{enumerate}
                \item Введение в геометрию
                \item Линии, углы, фигуры
                \item Площадь и периметр
                \item Практические задания
            \end{enumerate}
    \end{itemize}

\item \textbf{Тема:} Алгебра
    \begin{itemize}
        \item \textbf{Цель:} Освоить основы алгебры, решение уравнений и задач.
        \item \textbf{Порядок занятий:}
            \begin{enumerate}
                \item Введение в алгебру
                \item Решение уравнений
                \item Системы уравнений
                \item Практические задания
            \end{enumerate}
    \end{itemize}

\item \textbf{Тема:} Статистика и вероятность
    \begin{itemize}
        \item \textbf{Цель:} Понять основы статистики и вероятности, уметь анализировать данные.
        \item \textbf{Порядок занятий:}
            \begin{enumerate}
                \item Введение в статистику
                \item Построение гистограмм и диаграмм
                \item Основы вероятности
                \item Практические задания
            \end{enumerate}
    \end{itemize}

\end{enumerate}

\end{document}