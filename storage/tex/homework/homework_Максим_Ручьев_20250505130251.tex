Конечно, вот задание по следующей теме из учебного плана по предмету Математика для ученика Максима:

\documentclass{article}
\usepackage{amsmath}

\begin{document}

\title{Задание по математике}
\author{Ученик: Максим}
\date{}
\maketitle

\section*{Следующая тема: "Математические головоломки"}

\subsection*{Задача 1}
Решите головоломку "Задача о весах":
У вас есть 8 монет одинакового веса, но одна фальшивая монета отличается по весу (легче или тяжелее). Как за 2 взвешивания на чашечных весах без гирь определить, какая монета фальшивая и какого веса она?

\subsection*{Задача 2}
Попробуйте решить головоломку "Задача о кроликах и морковках":
На острове живут 100 кроликов и 100 морковок. Каждый кролик съедает одну морковку каждый день. За сколько дней на острове останется морковка?

\subsection*{Задача 3}
Найдите ответ на задачу "Загадка о числе 8":
Что такое число 8, если убрать из него все символы?

\subsection*{Задача 4}
Попробуйте разгадать загадку "Что это за число?":
Это число безопасно, но если добавить к нему 6, оно станет неустойчивым. Что это за число?

\subsection*{Задача 5}
Попробуйте решить головоломку "Задача о размене":
У вас есть 3 монеты достоинством 1 рубль, 3 монеты достоинством 2 рубля и 3 монеты достоинством 3 рубля. Как можно разменять 100 рублей, чтобы использовать все монеты?

\end{document}

Такие задачи позволят ученику Максиму развить логическое мышление и навыки решения нестандартных задач. Удачи в их решении!