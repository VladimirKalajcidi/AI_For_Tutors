Конечно, вот задание по следующей теме из учебного плана по предмету Математика для ученика Максима:

\documentclass{article}
\usepackage{amsmath}

\begin{document}

\title{Домашнее задание по Математике}
\author{Ученик: Максим}
\date{}
\maketitle

\section*{Следующая тема: "Сложные математические задачи"}

\subsection*{Задача 1}
Решите уравнение:
\[
3x - 5 = 2x + 7
\]

\subsection*{Задача 2}
Найдите произведение:
\[
(2x + 3)(x - 4)
\]

\subsection*{Задача 3}
Вычислите значение выражения:
\[
\frac{4x}{3} - \frac{x}{2} + 5
\]

\subsection*{Задача 4}
Решите систему уравнений:
\[
\begin{cases}
2x + y = 5 \\
x - 3y = -2
\end{cases}
\]

\subsection*{Задача 5}
Найдите значение выражения:
\[
\frac{2}{3} - \frac{1}{4} + \frac{5}{6}
\]

\end{document}

Эти задачи помогут Максиму развить навыки решения сложных математических задач. Успехов в выполнении!