\documentclass{article}
\usepackage{amsmath}

\begin{document}

\title{Домашнее задание по Математике}
\author{Ученик: Максим}
\date{}
\maketitle

\section*{Следующая тема: "Геометрические фигуры"}

\subsection*{Задача 1}
Найдите площадь круга с радиусом 10 единиц.

\subsection*{Задача 2}
Рассмотрим треугольник с углами 45°, 45°, 90°. Найдите длину гипотенузы, если катет равен 5 единицам.

\subsection*{Задача 3}
Дан равнобедренный треугольник с основанием 6 единиц и высотой 4 единицы. Найдите площадь этого треугольника.

\subsection*{Задача 4}
Рассмотрим прямоугольный параллелепипед с длиной 8 единиц, шириной 5 единиц и высотой 3 единицы. Найдите его объем.

\subsection*{Задача 5}
Найдите площадь правильного восьмиугольника, вписанного в окружность радиусом 6 единиц.

\end{document}