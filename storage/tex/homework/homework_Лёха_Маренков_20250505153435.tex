\documentclass{article}
\usepackage[utf8]{inputenc}
\usepackage{amsmath}

\begin{document}

\section*{Задание по математике для Лёхи, уровень: Богатые родители}

\subsection*{1. Покупка товаров}

Лёха с родителями отправился за покупками. Они купили 4 пары дорогих кроссовок по 3500 рублей за пару, 3 фирменных футболки по 2500 рублей за штуку и 2 дизайнерские сумки по 6000 рублей каждая. Сколько денег потратили Лёха и его родители?

\subsection*{2. Арифметика и логика}

Лёха решил загадать своему другу загадку. Он сказал: "Если 5 пингвинов держат 3 зонтика, то сколько зонтиков понадобится для 15 пингвинов?" Помогите другу Лёхи решить эту загадку.

\subsection*{3. Геометрия}

Лёха построил параллелограмм со сторонами 8 см и 12 см. Найдите площадь этого параллелограмма.

\subsection*{4. Уравнения}

Лёха решил уравнение: $2x^2 - 5x + 3 = 0$. Найдите корни этого уравнения.

\subsection*{5. Задача на логику}

Лёха и его друг Вася играют в игру. Они загадали число от 1 до 100. Лёха сказал: "Моё число нечётное". Вася сказал: "Моё число делится на 3". Какое число угадал Лёха, а какое Вася?

\end{document}