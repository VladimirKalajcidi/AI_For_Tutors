\documentclass{article}
\usepackage[utf8]{inputenc}
\usepackage[russian]{babel}
\usepackage{geometry}
\geometry{a4paper, margin=25mm}
\begin{document}

Домашнее задание по математике для ученика Вадима:

\[
\begin{array}{|c|l|}
\hline
1. & \text{Задача про Пуха и его горшок меда: Винни Пух съедает 8 горшков меда в день. Сколько горшков меда он съест за 5 дней?} \\
2. & \text{Задача про Тигрулю и его хвост: Если хвост Тигрули длиной 12 см, а это в 3 раза длиннее хвоста Пятачка, какой длины хвост у Пятачка?} \\
3. & \text{Задача про Иа-Иа и его уши: Если у Иа-Иа 3 пары ушей, а у каждой пары 2 уша, сколько всего ушей у Иа-Иа?} \\
4. & \text{Задача про Кролика и морковку: Кролик съедает 1 морковку за 15 минут. Сколько морковок он съест за 1 час?} \\
5. & \text{Задача про Сову и книги: Если Сова прочитывает 4 книги в месяц, то за год сколько книг она прочитает?} \\
6. & \text{Задача про Пятачка и пчёл: Пятачок насчитал 10 пчёл на одном цветке, а на другом вдвое меньше. Сколько всего пчёл он увидел?} \\
7. & \text{Задача про Кролика и овощи: Если Кролик собрал 7 морковок и 4 редьки, сколько всего овощей у него получилось?} \\
8. & \text{Задача про Винни Пуха и пчелиную ульицу: Если Винни Пух насчитал 6 ульиц с медом, а в каждой ульице по 3 горшка меда, сколько горшков меда у него всего?} \\
9. & \text{Задача про Кролика и его друзей: Если у Кролика 5 друзей, а у каждого друга по 2 морковки, сколько морковок всего у Кролика и его друзей?} \\
10. & \text{Задача про Пятачка и его деньги: Если у Пятачка было 15 монеток, а он потратил половину, сколько монеток у него осталось?} \\
\hline
\end{array}
\]

Желаю Вадиму удачи в решении задач!
\end{document}