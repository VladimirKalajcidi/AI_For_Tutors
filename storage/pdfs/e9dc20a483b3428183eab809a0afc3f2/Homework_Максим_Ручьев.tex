\documentclass{article}
\usepackage[utf8]{inputenc}
\usepackage[russian]{babel}
\usepackage{geometry}
\geometry{a4paper, margin=25mm}
\begin{document}

Конечно, вот обновленное домашнее задание:

\documentclass{article}
\usepackage{amsmath}

\begin{document}

\title{Домашнее задание по Математике}
\author{Ученик: Максим}
\date{}
\maketitle

\section*{Следующая тема: "Геометрические фигуры"}

\subsection*{Задача 1}
Найдите площадь треугольника со сторонами 7, 8 и 9 единиц.

\subsection*{Задача 2}
Найдите объем цилиндра, если его высота равна 10 единицам, а радиус основания 5 единиц.

\subsection*{Задача 3}
Рассмотрим равносторонний треугольник со стороной 12 единиц. Найдите его площадь.

\subsection*{Задача 4}
Дан прямоугольник со сторонами 6 и 10 единиц. Найдите его периметр.

\subsection*{Задача 5}
Найдите объем пирамиды, если ее высота равна 8 единицам, а площадь основания 24 квадратных единицы.

\end{document}
\end{document}