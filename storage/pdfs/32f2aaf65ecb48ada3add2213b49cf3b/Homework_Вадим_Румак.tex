\documentclass{article}
\usepackage[utf8]{inputenc}
\usepackage[russian]{babel}
\usepackage{geometry}
\geometry{a4paper, margin=25mm}
\begin{document}

Конечно, вот домашнее задание по математике для ученика Вадима:

\[
\text{Домашнее задание по математике для ученика Вадима}
\]

\textbf{Часть 1: Арифметика}

1. Решите уравнение: \(5x + 3 = 18\).
2. Посчитайте \(3 \times \frac{2}{5}\).
3. Найдите среднее арифметическое чисел 8, 12, 15, 21.
4. Решите задачу: Если 20% числа равно 30, то какое это число?
5. Сложите дроби: \(\frac{3}{4} + \frac{2}{3}\).

\textbf{Часть 2: Геометрия}

6. Найдите площадь прямоугольника со сторонами 5 см и 8 см.
7. Решите задачу: Площадь треугольника равна 24 кв. см, а его высота равна 6 см. Найдите основание треугольника.
8. Найдите длину окружности, если её радиус равен 7 см.
9. Сколько градусов в 1/4 оборота?
10. Найдите объём параллелепипеда, если его длина 4 см, ширина 6 см, высота 10 см.

\textbf{Часть 3: Алгебра}

11. Разложите на множители: \(x^2 + 5x + 6\).
12. Решите систему уравнений: \(\begin{cases} 2x + y = 5 \\ x - y = 1 \end{cases}\).
13. Решите уравнение: \(2^{x-1} = 8\).
14. Найдите значение выражения при \(x = 2\): \(3x^2 - 4x + 5\).
15. Постройте график функции \(y = 2x + 3\).

\textbf{Часть 4: Сложные случаи}

16. Решите квадратное уравнение: \(x^2 - 7x + 12 = 0\).
17. Найдите сумму бесконечно убывающей геометрической прогрессии с первым членом 8 и знаменателем 1/2.
18. Решите задачу: В треугольнике ABC угол A равен 50 градусов, а сторона BC равна 12 см. Найдите сторону AC.
19. Найдите значение выражения \(\frac{5!}{3!}\).
20. Решите уравнение: \(\sqrt{3x + 1} = 5\).

\textbf{Часть 5: Ловушки}

21. Какая из дробей больше: \(\frac{3}{5}\) или \(\frac{4}{7}\)?
22. Найдите ошибку в решении уравнения: \(2x - 5 = 11 \Rightarrow x = \frac{11}{2} + 5\).
23. Решите задачу: Если 1 яблоко стоит 10 рублей, то сколько стоит 5 яблок?
24. Найдите сумму ряда: \(1 + \frac{1}{2} + \frac{1}{4} + \frac{1}{8} + \ldots\).
25. Если \(a = 3\) и \(b = 2\), то что больше: \(a^b\) или \(b^a\)?

\textbf{Часть 6: Открытые вопросы}

26. Какие свойства имеют параллелограммы?
27. Чем отличаются правильный и произвольный многоугольники?
28. Какие способы решения уравнений существуют?
29. В чём отличие линейной и квадратичной функций?
30. Какое значение имеет число π в математике?

\textbf{Бонусный вопрос}

31. Какие темы в математике тебе кажутся наиболее интересными или сложными?

Желаю Вадиму успешно выполнить задания и улучшить свои знания!
\end{document}