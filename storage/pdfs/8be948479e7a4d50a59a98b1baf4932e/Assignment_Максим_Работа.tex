\documentclass{article}
\usepackage[utf8]{inputenc}
\usepackage[russian]{babel}
\usepackage{geometry}
\geometry{a4paper, margin=25mm}
\begin{document}

```latex
\documentclass[a4paper,12pt]{article}
\usepackage{amsmath}
\usepackage{amssymb}

\begin{document}

\title{Задание по теме: Термодинамика}
\author{Для ученика Максим, уровень: Хороший парень, старательный}
\date{}
\maketitle

\vspace{0.5cm}

\textbf{Инструкция:} Решите следующую задачу по теме "Термодинамика". Внимательно проанализируйте условие, используйте основные законы термодинамики и при необходимости сделайте обоснованные допущения.

\vspace{0.5cm}

\begin{enumerate}

\item \textbf{Адиабатический процесс с изменяющимся объёмом.} \\
Идеальный газ в цилиндре с поршнем изначально находится в состоянии с объёмом 
\[
V_1 = 2\,\mathrm{л}
\]
и температурой 
\[
T_1 = 300\,\mathrm{K}.
\] 
Газ расширяется адиабатически так, что объём увеличивается в 4 раза. Найдите конечную температуру газа $T_2$, если показатель адиабаты 
\[
\gamma = 1.4.
\]

\end{enumerate}

\vspace{0.5cm}

\textbf{Желаю успехов!}

\end{document}
```
\end{document}