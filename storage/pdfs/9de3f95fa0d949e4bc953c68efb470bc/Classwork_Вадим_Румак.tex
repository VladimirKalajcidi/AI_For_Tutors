\documentclass{article}
\usepackage[utf8]{inputenc}
\usepackage[russian]{babel}
\usepackage{geometry}
\geometry{a4paper, margin=25mm}
\begin{document}

Конечно, вот контрольная работа по предмету Математика для ученика Вадима:

\[
\text{=== Контрольная работа по математике для ученика Вадима ===}
\]

\[
\text{**Часть 1: Арифметика**}
\]

\[
\begin{array}{|c|l|}
\hline
1. & \text{Решите уравнение: } 5x + 3 = 18 \\
2. & \text{Посчитайте } 3 \times \frac{2}{5} \\
3. & \text{Найдите среднее арифметическое чисел 8, 12, 15, 21} \\
4. & \text{Решите задачу: Если 20\% числа равно 30, то какое это число?} \\
5. & \text{Сложите дроби: } \frac{3}{4} + \frac{2}{3} \\
\hline
\end{array}
\]

\[
\text{**Часть 2: Геометрия**}
\]

\[
\begin{array}{|c|l|}
\hline
6. & \text{Найдите площадь прямоугольника со сторонами 5 см и 8 см} \\
7. & \text{Решите задачу: Площадь треугольника равна 24 кв. см, а его высота равна 6 см. Найдите основание треугольника.} \\
8. & \text{Найдите длину окружности, если её радиус равен 7 см} \\
9. & \text{Сколько градусов в 1/4 оборота?} \\
10. & \text{Найдите объём параллелепипеда, если его длина 4 см, ширина 6 см, высота 10 см} \\
\hline
\end{array}
\]

\[
\text{**Часть 3: Алгебра**}
\]

\[
\begin{array}{|c|l|}
\hline
11. & \text{Разложите на множители: } x^2 + 5x + 6 \\
12. & \text{Решите систему уравнений: } \begin{cases} 2x + y = 5 \\ x - y = 1 \end{cases} \\
13. & \text{Решите уравнение: } 2^{x-1} = 8 \\
14. & \text{Найдите значение выражения при } x = 2: 3x^2 - 4x + 5 \\
15. & \text{Постройте график функции } y = 2x + 3 \\
\hline
\end{array}
\]

\[
\text{**Часть 4: Сложные случаи**}
\]

\[
\begin{array}{|c|l|}
\hline
16. & \text{Решите квадратное уравнение: } x^2 - 7x + 12 = 0 \\
17. & \text{Найдите сумму бесконечно убывающей геометрической прогрессии с первым членом 8 и знаменателем 1/2} \\
18. & \text{Решите задачу: В треугольнике ABC угол A равен 50 градусов, а сторона BC равна 12 см. Найдите сторону AC} \\
19. & \text{Найдите значение выражения } \frac{5!}{3!} \\
20. & \text{Решите уравнение: } \sqrt{3x + 1} = 5 \\
\hline
\end{array}
\]

\[
\text{**Часть 5: Ловушки**}
\]

\[
\begin{array}{|c|l|}
\hline
21. & \text{Какая из дробей больше: } \frac{3}{5} \text{ или } \frac{4}{7}? \\
22. & \text{Найдите ошибку в решении уравнения: } 2x - 5 = 11 \Rightarrow x = \frac{11}{2} + 5 \\
23. & \text{Решите задачу: Если 1 яблоко стоит 10 рублей, то сколько стоит 5 яблок?} \\
24. & \text{Найдите сумму ряда: } 1 + \frac{1}{2} + \frac{1}{4} + \frac{1}{8} + \ldots \\
25. & \text{Если } a = 3 \text{ и } b = 2, \text{ то что больше: } a^b \text{ или } b^a? \\
\hline
\end{array}
\]

\[
\text{**Часть 6: Открытые вопросы**}
\]

\[
\begin{array}{|c|l|}
\hline
26. & \text{Какие свойства имеют параллелограммы?} \\
27. & \text{Чем отличаются правильный и произвольный многоугольники?} \\
28. & \text{Какие способы решения уравнений существуют?} \\
29. & \text{В чём отличие линейной и квадратичной функций?} \\
30. & \text{Какое значение имеет число } \pi \text{ в математике?} \\
\hline
\end{array}
\]

\[
\text{**Бонусный вопрос**}
\]

\[
31. \text{ Какие темы в математике тебе кажутся наиболее интересными или сложными?}
\]
\end{document}