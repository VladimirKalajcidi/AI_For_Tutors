\documentclass{article}
\usepackage[utf8]{inputenc}
\usepackage[russian]{babel}
\usepackage{geometry}
\geometry{a4paper, margin=25mm}
\begin{document}

Данное домашнее задание содержит разнообразные задачи по математике для ученика Максима на уровне "Учиться в сунц нгу". При необходимости, я готов помочь с решением задач или дополнительными пояснениями.

\documentclass{article}
\usepackage[utf8]{inputenc}
\usepackage{amsmath}

\begin{document}

\section*{Домашнее задание}

\subsection*{Задачи}

\begin{enumerate}
    \item Найдите значение выражения:
    \begin{enumerate}
        \item $\sqrt{625} - 5 \times 3$
        \item $2^4 + 3 \times 7$
    \end{enumerate}

    \item Решите уравнение:
    \begin{enumerate}
        \item $4x + 9 = 25$
        \item $\dfrac{2}{3}x = 8$
    \end{enumerate}

    \item Вычислите:
    \begin{enumerate}
        \item Площадь треугольника со сторонами 6, 8 и 10 (по формуле Герона).
        \item Результат выражения: $\dfrac{5^2 + 3 \times 4}{2}$
    \end{enumerate}

    \item Найдите корень уравнения: $x^2 - 7x + 10 = 0$

    \item Решите систему уравнений:
    \begin{align*}
        2x + y &= 7 \\
        x - 3y &= 11
    \end{align*}
\end{enumerate}

\subsection*{Дополнительные задачи}

\begin{enumerate}
    \item Вычислите площадь квадрата, описанного вокруг круга радиусом 5 единиц.
    
    \item Решите уравнение: $3 \sqrt{x} = 12$
    
    \item Найдите значение выражения: $\dfrac{3^3 - 2^4}{5}$
    
    \item Найдите периметр правильного шестиугольника, вписанного в окружность радиусом 10 единиц.
    
    \item Решите задачу: В треугольнике угол при основании равен 60 градусов, а сторона, противолежащая углу, равна 8 единиц. Найдите площадь треугольника.

\end{enumerate}

\end{document}
\end{document}