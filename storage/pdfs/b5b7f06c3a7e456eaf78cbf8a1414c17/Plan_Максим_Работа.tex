\documentclass{article}
\usepackage[utf8]{inputenc}
\usepackage[russian]{babel}
\usepackage{geometry}
\geometry{a4paper, margin=25mm}
\begin{document}

Вот исправленный учебный план по предмету Физика для Максима:

\documentclass{article}
\usepackage{enumitem}

\begin{document}

\title{\textbf{Учебный план по Физике для Максима}}
\author{}
\date{}
\maketitle

\section*{Уровень: Хороший парень, старательный}

\section*{Тема: Термодинамика}

\begin{enumerate}[label=\textbf{Урок \arabic*:}, leftmargin=*]
  \item \textbf{Первый закон термодинамики и его применение}
    \begin{itemize}
        \item Цель: Разбор первого закона термодинамики и решение задач по нему.
        \item Задачи: Связь с законом сохранения энергии и законом сохранения импульса.
    \end{itemize}
  
  \item \textbf{Циклы тепловых машин}
    \begin{itemize}
        \item Цель: Изучение циклов тепловых машин и их применение.
        \item Задачи: Решение задач по циклам.
    \end{itemize}
  
  \item \textbf{Электрические цепи: вычисление параметров и свойства}
    \begin{itemize}
        \item Цель: Изучение электрических цепей и вычисление сопротивления, тока и других параметров.
        \item Задачи: Решение задач по вычислению параметров цепей.
    \end{itemize}

  \item \textbf{Законы сохранения энергии в электрических цепях}
    \begin{itemize}
        \item Цель: Понимание законов сохранения энергии в электрических цепях.
        \item Задачи: Решение задач по закону сохранения энергии.
    \end{itemize}

  \item \textbf{Кулоновское взаимодействие}
    \begin{itemize}
        \item Цель: Изучение явления кулоновского взаимодействия и решение задач.
        \item Задачи: Решение задач по кулоновскому взаимодействию.
    \end{itemize}

  % Продолжайте добавлять уроки согласно плану

\end{enumerate}

\end{document} 

Уроки структурированы с указанием целей и задач для каждого урока по темам, которые вы предложили. План сохраняет стиль и структуру документа в LaTeX-стиле. Добавляйте дополнительные уроки по мере продвижения по плану обучения Максима.
\end{document}