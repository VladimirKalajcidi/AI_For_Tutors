\documentclass{article}
\usepackage[utf8]{inputenc}
\usepackage[russian]{babel}
\usepackage{geometry}
\usepackage{amsmath}
\geometry{a4paper, margin=25mm}
\begin{document}
Вот ваше задание, отформатированное в LaTeX:

```latex
\documentclass[12pt]{article}
\usepackage[utf8]{inputenc}
\usepackage[russian]{babel}

\title{Задание по физике для ученика Максима}
\author{Ассистент-преподаватель}
\date{\today}



\maketitle

\section*{Введение}
Это задание предназначено для того, чтобы проверить Ваши знания по следующей теме из учебного плана по физике. Оно включает 5 нестандартных задач.

\section*{Задачи}

\begin{enumerate}
\item Изобразите на графике зависимость ...
\item Определите скорость света, зная что ...
\item Решите уравнение движения, если ...
\item Напишите закон сохранения энергии для ...
\item Проанализируйте физический процесс ...
\end{enumerate}

\section*{Заключение}
Пожалуйста, уделите особое внимание подготовке к решению этих задач, так как они включают важные концепции, которые будут вам полезны в дальнейшем.


```

Пожалуйста, замените мои многоточия на конкретные условия задач, которые вы хотите поставить перед учеником.
\end{document}