\documentclass{article}
\usepackage[utf8]{inputenc}
\usepackage[russian]{babel}
\usepackage{geometry}
\geometry{a4paper, margin=25mm}
\begin{document}

Конечно, вот обновленный учебный план по предмету Математика для ученика Максим:

```latex
\documentclass{article}
\usepackage[utf8]{inputenc}

\begin{document}

\title{Учебный план по Математике для Максима}
\author{Преподаватель: Имя Фамилия}
\date{Год}

\maketitle

\section{Цели}
\begin{itemize}
    \item Познакомить ученика с основными математическими понятиями и методами решения задач.
    \item Развить логическое мышление и абстрактное мышление ученика.
    \item Подготовить ученика к изучению более сложных математических тем в будущем.
\end{itemize}

\section{План занятий}

\subsection{Тема: Числа и действия над ними}
\begin{itemize}
    \item Цели: Познакомить ученика с основными свойствами чисел, основными арифметическими операциями.
    \item Занятие 1: Введение в натуральные числа, сложение и вычитание.
    \item Занятие 2: Умножение и деление натуральных чисел.
    \item Занятие 3: Введение в целые числа, операции с целыми числами.
\end{itemize}

\subsection{Тема: Геометрия}
\begin{itemize}
    \item Цели: Ознакомить ученика с основными геометрическими фигурами и их свойствами.
    \item Занятие 1: Основные понятия геометрии, линии, отрезки, углы.
    \item Занятие 2: Треугольники, прямоугольники, круги.
    \item Занятие 3: Площади фигур, периметр.
\end{itemize}

\subsection{Тема: Уравнения и неравенства}
\begin{itemize}
    \item Цели: Обучить ученика решению уравнений и неравенств.
    \item Занятие 1: Основные типы уравнений и их решение.
    \item Занятие 2: Уравнения с одной и двумя переменными.
    \item Занятие 3: Неравенства, их свойства и решение.
\end{itemize}

\section{Итоговая проверка}
Провести контрольную работу, включающую задачи по всем изученным темам.

\end{document}
```
\end{document}