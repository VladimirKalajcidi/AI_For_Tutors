\documentclass{article}
\usepackage[utf8]{inputenc}
\usepackage[russian]{babel}
\usepackage{geometry}
\geometry{a4paper, margin=25mm}
\begin{document}

1. Простые арифметические действия:
   a) 25 + 14 =
   b) 36 - 19 =
   c) 7 * 8 =
   d) 45 / 5 =

2. Решение уравнений:
   a) 3x + 5 = 14, найди x.
   b) 2y - 10 = 16, найди y.

3. Геометрия:
   a) Чему равен периметр прямоугольника со сторонами 6 см и 10 см?
   b) Найди площадь круга с радиусом 5 см.

4. Дроби и проценты:
   a) Переведи 3/5 в проценты.
   b) 25% от числа 80 равно...

5. Сложные задачи:
   a) На тренировку пришло 36 человек. Если 1/4 из них были девочки, сколько всего мальчиков пришло на тренировку?
   b) Тетя Марина купила 3 кг яблок по 60 рублей за килограмм, а также 2 кг апельсинов по 40 рублей за килограмм. Сколько денег она потратила?

6. Логика и математические рассуждения:
   a) Если a = 5 и b = 3, то что будет результатом выражения 2a + 3b?
   b) Если все кошки любят молоко, а Мурка - кошка, значит ли это, что Мурка любит молоко?

7. «Ловушки»:
   a) Чему равно 0! (факториал нуля)?
   b) Чему равно корень из -9?

8. Бонусный вопрос:
   a) Какое правило позволяет упростить выражение 3x + 2x?

Желаю удачи на тесте!
\end{document}