\documentclass{article}
\usepackage[utf8]{inputenc}
\usepackage[russian]{babel}
\usepackage{geometry}
\geometry{a4paper, margin=25mm}
\begin{document}

Конечно, вот отчёт о успеваемости ученика Вадима по предмету Математика:

```latex
\documentclass{article}
\usepackage{amsmath}

\begin{document}

\title{Отчёт об успеваемости ученика Вадима по математике}
\author{Родителям}
\date{\today}
\maketitle

\section{Результаты диагностического теста}

\subsection{Арифметика}
\begin{enumerate}
    \item \(5x + 3 = 18\): \textbf{Неверно}
    \item \(3 \times \frac{2}{5}\): \textbf{Неверно}
    \item Среднее арифметическое: \textbf{Неверно}
    \item 20\% числа: \textbf{Неверно}
    \item \(\frac{3}{4} + \frac{2}{3}\): \textbf{Неверно}
\end{enumerate}

\subsection{Геометрия}
\begin{enumerate}
    \item Площадь прямоугольника: \textbf{Неверно}
    \item Основание треугольника: \textbf{Неверно}
    \item Длина окружности: \textbf{Неверно}
    \item Градусы в 1/4 оборота: \textbf{Неверно}
    \item Объём параллелепипеда: \textbf{Неверно}
\end{enumerate}

\subsection{Алгебра}
\begin{enumerate}
    \item Разложение на множители: \textbf{Неверно}
    \item Система уравнений: \textbf{Неверно}
    \item \(2^{x-1} = 8\): \textbf{Неверно}
    \item \(3x^2 - 4x + 5\), при \(x = 2\): \textbf{Неверно}
    \item График функции \(y = 2x + 3\): \textbf{Неверно}
\end{enumerate}

\subsection{Сложные случаи}
\begin{enumerate}
    \item Квадратное уравнение: \textbf{Неверно}
    \item Сумма геометрической прогрессии: \textbf{Неверно}
    \item Сторона треугольника: \textbf{Неверно}
    \item \(\frac{5!}{3!}\): \textbf{Неверно}
    \item \(\sqrt{3x + 1} = 5\): \textbf{Неверно}
\end{enumerate}

\subsection{Ловушки}
\begin{enumerate}
    \item Сравнение дробей: \textbf{Неверно}
    \item Ошибка в решении уравнения: \textbf{Неверно}
    \item Стоимость яблок: \textbf{Неверно}
    \item Сумма ряда: \textbf{Неверно}
    \item Сравнение степеней: \textbf{Неверно}
\end{enumerate}

\section{Рекомендации}

Учитывая результаты теста, рекомендуется Вадиму уделить больше внимания учёбе, активнее участвовать в уроках и выполнении домашних заданий. Также важно ограничить время, проведённое в социальных сетях, чтобы улучшить успеваемость и понимание материала.

\section{Бонусный вопрос}

\textbf{Какие темы в математике тебе кажутся наиболее интересными или сложными?}

\end{document}
```

Предложенный отчёт подробно описывает результаты диагностического теста по математике для ученика Вадима и дает рекомендации по улучшению учебного процесса.
\end{document}