\documentclass{article}
\usepackage[utf8]{inputenc}
\usepackage[russian]{babel}
\usepackage{geometry}
\geometry{a4paper, margin=25mm}
\begin{document}

Хорошо, вот масштабный диагностический тест по математике для ученика:

Часть 1: Базовые понятия

1. Что такое наибольшее общее кратное (НОК) двух чисел?
2. Определите результат умножения: \((-3) \times 7\).
3. Какие из следующих чисел являются простыми: 15, 23, 31, 42?
4. Что такое десятичная дробь?
5. Решите уравнение: \(2x + 5 = 17\).

Часть 2: Алгебра

6. Решите систему уравнений:
   \[
   \begin{cases}
   2x + y = 5 \\
   x - y = 1
   \end{cases}
   \]
7. Факторизуйте выражение: \(x^2 - 4\).
8. Какие из данных чисел являются корнями уравнения \(x^2 - 5x + 6 = 0\): 2, 3, 4, 6?
9. Найдите значение выражения при \(x = 3\): \(2x^2 - 5x + 3\).
10. Решите неравенство: \(-3x + 7 \geq 13\).

Часть 3: Геометрия

11. Чем отличаются прямоугольник и квадрат?
12. Найдите периметр треугольника с длинами сторон 4, 5 и 6.
13. Чему равна сумма углов треугольника?
14. Как найти площадь круга, если известен его радиус?
15. Что такое синус угла в прямоугольном треугольнике?

Часть 4: Смешанные задачи

16. Каковы корни уравнения \(x^2 - 6x + 9 = 0\) и как это связано с геометрией?
17. Решите задачу: Если сторона квадрата увеличена в 3 раза, то как изменится его площадь?
18. Какие числа можно вставить вместо a и b в выражение \(a^2 - b^2\) для получения тождества?
19. Найдите среднее арифметическое чисел 4, 7 и 10.
20. Какую формулу использовать для нахождения объема параллелепипеда?

Этот тест должен помочь выявить пробелы в знаниях ученика и помочь ему в дальнейшем улучшить свои математические навыки.
\end{document}