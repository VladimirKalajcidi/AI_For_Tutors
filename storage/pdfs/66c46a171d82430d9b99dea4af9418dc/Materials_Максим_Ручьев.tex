\documentclass{article}
\usepackage[utf8]{inputenc}
\usepackage[russian]{babel}
\usepackage{geometry}
\geometry{a4paper, margin=25mm}
\begin{document}

Конечно, вот обновленный учебный материал:

\documentclass{article}
\usepackage{enumitem}

\begin{document}

\title{Математика для Максима}
\author{Преподаватель}
\date{\today}
\maketitle

\section{Тема 1: Алгебра}

\subsection{Понятие алгебраического выражения}

Алгебраическое выражение – это математическое выражение, в котором могут присутствовать переменные, числа и операции сложения, вычитания, умножения и деления.

\subsection{Уравнения и неравенства}

\begin{enumerate}
\item Решение уравнений. Для решения уравнений используйте свойства равенств и преобразования уравнений.
\item Решение неравенств. Помните, что при умножении или делении на отрицательное число знак неравенства меняется.
\end{enumerate}

\subsection{Книги для чтения}

\begin{itemize}
\item "Алгебра. Начальный курс" автора А.И. Козлов
\item "Учебник алгебры" под редакцией И.И. Гельфанд
\end{itemize}

\section{Тема 2: Геометрия}

\subsection{Основные понятия}

\begin{itemize}
\item Прямые и углы.
\item Треугольники.
\item Четырёхугольники.
\end{itemize}

\subsection{Площади фигур}

\begin{enumerate}
\item Формулы для нахождения площадей прямоугольника, треугольника и круга.
\item Задачи на нахождение площадей сложных фигур.
\end{enumerate}

\subsection{Книги для чтения}

\begin{itemize}
\item "Геометрия. Учебник для 7-9 классов" автора А.П. Кожевников
\item "Введение в геометрию" под редакцией И.А. Тамм
\end{itemize}

\section{Заключение}

Математика — увлекательный предмет, который помогает развивать логическое мышление и аналитические способности. Постоянно практикуйтесь и не бойтесь новых задач!

\end{document}
\end{document}