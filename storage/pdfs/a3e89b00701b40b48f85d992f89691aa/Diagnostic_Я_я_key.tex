\documentclass{article}
\usepackage[utf8]{inputenc}
\usepackage[russian]{babel}
\usepackage{geometry}
\geometry{a4paper, margin=25mm}
\begin{document}

Конечно, вот подробный ключ ответов к предложенному масштабному диагностическому тесту по математике:

```latex
\documentclass{article}
\usepackage{amsmath}

\begin{document}

\section*{Часть 1: Основные понятия}

1. Действительные числа — это числа, которые можно представить на числовой прямой. Пример: \( \mathbb{R} \).
2. Дробь — это отношение двух целых чисел, например, \( \frac{3}{4} \). Примеры действий: сложение: \( \frac{1}{3} + \frac{1}{6} = \frac{1}{2} \); умножение: \( \frac{2}{5} \times \frac{3}{4} = \frac{6}{20} = \frac{3}{10} \).
3. Четность или нечетность числа определяется его остатком от деления на 2.
4. Простые числа имеют только два делителя: 1 и само число. Пример: 7. Составные числа имеют более двух делителей. Пример: 12.
5. Свойства операций: коммутативность, ассоциативность, дистрибутивность, нейтральные элементы и обратные элементы.

\section*{Часть 2: Алгебраические выражения}

6. Алгебраическое выражение — выражение, содержащее переменные, константы и операции. Пример: \( 2x + 3y \).
7. Упрощение алгебраических выражений — это уменьшение выражения до более простого вида.
8. Решение уравнения: \(3x + 5 = 17\) даёт \(x = 4\).
9. Уравнения утверждают равенство двух выражений, неравенства устанавливают отношение между ними.
10. Для решения системы уравнений используют метод подстановки, метод исключения, метод графический и др.

\section*{Часть 3: Геометрия}

11. Виды углов: прямой, тупой, острый. Пример: прямой угол \(90^\circ\).
12. Параллельные прямые не пересекаются, перпендикулярные образуют прямой угол.
13. Площадь треугольника вычисляется по формуле \( S = \frac{1}{2} \times \text{основание} \times \text{высота} \).
14. Прямоугольник — четырехугольник с противоположными сторонами, квадрат — прямоугольник с равными сторонами, ромб — четырехугольник с равными сторонами.
15. Объем цилиндра: \( V = \pi r^2 h \), площадь поверхности: \( S = 2\pi r^2 + 2\pi r h \).

\section*{Часть 4: Сложные задачи}

16. Пусть \(x\) — кг яблок, \(y\) — кг апельсинов. Система уравнений: \(x + y = 2\), \(15x + 25y = 60\). Решение: \(x = \frac{4}{5} \), \(y = \frac{2}{5} \).
17. Используем выражения суммы и произведения корней уравнения для нахождения \(a\) и \(b\): \(a = 2\), \(b = 7\).
18. Поезд преодолевает 240 км со скоростью 60 км/ч, значит, через 4 часа.
19. Сумма членов прогрессии: \(4 + 7 + 10 + \ldots + 28 = 220\).
20. Признаки простоты: делится только на 1 и само число, больше 1.

\section*{Часть 5: Доказательства и Обоснования}

21. Сумма углов треугольника равна 180 градусов, так как угол в равнобедренном треугольнике равен углу при основании.
22. Умножение на ноль дает ноль, так как это связано с арифметическим свойством нуля.
23. Корень квадратный из числа неотрицателен, так как он представляет длину стороны квадрата.
24. Метод подстановки используется для замены переменных в уравнении.
25. Произведение двух четных чисел всегда четное, так как четное число делится на 2.

\section*{Часть 6: Открытые вопросы}

26. Сложными темами в математике могут быть анализ, теория вероятностей, дифференциальные уравнения и др.
27. Методы решения задач зависят от предпочтений ученика: аналитический, графический, численный и др.
28. Интересные области математики для изучения могут быть геометрия, алгебра, теория чисел, математическая логика и др.

\end{document}
```
\end{document}