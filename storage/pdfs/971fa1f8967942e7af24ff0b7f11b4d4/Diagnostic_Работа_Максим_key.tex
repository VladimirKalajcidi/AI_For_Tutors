\documentclass{article}
\usepackage[utf8]{inputenc}
\usepackage[russian]{babel}
\usepackage{geometry}
\geometry{a4paper, margin=25mm}
\begin{document}

Да, конечно, вот подробный ключ ответов к диагностическому тесту по физике для ученика Максим:

```latex
\documentclass{article}
\usepackage{enumitem}

\begin{document}

\section*{Диагностический тест по физике для ученика Максим}

\subsection*{Часть 1. Механика}

\begin{enumerate}[label={\arabic*.}]
    \item Отличается скорость от ускорения:
    \begin{itemize}
        \item[a)] Скорость - векторная величина, ускорение - скалярная
    \end{itemize}
    
    \item Ускорение свободного падения на Земле равно:
    \begin{itemize}
        \item[a)] 9,8 м/с²
    \end{itemize}
    
    \item Формула для расчета пути равноускоренного движения:
    \begin{itemize}
        \item[a)] $s = v_0t + \frac{1}{2}at^2$
    \end{itemize}
\end{enumerate}

\subsection*{Часть 2. Термодинамика}

\begin{enumerate}[resume]
    \item Адиабатический процесс:
    \begin{itemize}
        \item[b)] Процесс, при котором в системе нет теплообмена с окружающей средой
    \end{itemize}
    
    \item Закон сохранения энергии формулируется в:
    \begin{itemize}
        \item[c)] Закон Гесса
    \end{itemize}
    
    \item Теплоемкость:
    \begin{itemize}
        \item[a)] Количество теплоты, необходимое для нагревания вещества на 1 градус Цельсия
    \end{itemize}
\end{enumerate}

\subsection*{Часть 3. Электричество и магнетизм}

\begin{enumerate}[resume]
    \item Формула для вычисления силы тока в цепи:
    \begin{itemize}
        \item[a)] $I = V/R$
    \end{itemize}
    
    \item Электрическое поле:
    \begin{itemize}
        \item[a)] Область пространства, в которой действует электрическая сила
    \end{itemize}
    
    \item Явление изменения направления тока в проводнике при изменении магнитного поля называется:
    \begin{itemize}
        \item[a)] Электромагнитная индукция
    \end{itemize}
\end{enumerate}

\subsection*{Часть 4. Оптика}

\begin{enumerate}[resume]
    \item Полное внутреннее отражение света:
    \begin{itemize}
        \item[a)] Явление, при котором свет не проникает из оптически более плотной среды в менее плотную
    \end{itemize}
    
    \item Формула для определения фокусного расстояния линзы:
    \begin{itemize}
        \item[a)] $1/f = 1/d_0 + 1/d_i$
    \end{itemize}
    
    \item Дисперсия света:
    \begin{itemize}
        \item[b)] Разложение света на спектральные составляющие
    \end{itemize}
\end{enumerate}

\subsection*{Часть 5. Атомная и ядерная физика}

\begin{enumerate}[resume]
    \item Протон является:
    \begin{itemize}
        \item[c)] Фермион
    \end{itemize}
    
    \item Радиоактивность:
    \begin{itemize}
        \item[a)] Способность вещества испускать радиоактивное излучение
    \end{itemize}
    
    \item Процесс деления ядра атома на более легкие ядра называется:
    \begin{itemize}
        \item[г)] Ядерное деление
    \end{itemize}
\end{enumerate}

\end{document}
```

Скопируйте данный код и вставьте его в свой LaTeX-документ для получения подробного ключа ответов к диагностическому тесту по физике для Максима.
\end{document}