\documentclass{article}
\usepackage[utf8]{inputenc}
\usepackage[russian]{babel}
\usepackage{geometry}
\geometry{a4paper, margin=25mm}
\begin{document}

Конечно, вот задание по следующей теме из учебного плана по математике для ученика Максима:

\documentclass{article}
\usepackage{amsmath}

\begin{document}

\title{Задание по математике}
\author{Максим}
\date{}
\maketitle

\section*{Тема: Нестандартные задачи}

\begin{enumerate}
    \item Известно, что сумма двух чисел равна 12, а их произведение равно 35. Найдите эти числа.
    
    \item В саду растут яблони и груши. Если собрать яблоки и груши вместе, то получится 40 штук. Однако, если яблоки раздать среди собравшихся, каждый получит по 5 яблок, а если груши раздать, то каждый получит по 3 груши. Сколько яблок и груш было изначально?
    
    \item У Максима есть коллекция марок. Если сложить количество марок, умножить на 3 и вычесть 5, то получится 22. Сколько марок у Максима?
    
    \item В классе 30 учеников. Каждый ученик купил как минимум одну книгу. Известно, что если разделить количество купленных книг на 5, то получится остаток 3. Сколько книг в среднем купил каждый ученик?
    
    \item Петя и Вася играли в шахматы. Если Петя выигрывал, то он получал 5 рублей, а если проигрывал, то отдавал 3 рубля Васе. После игры у Пети осталось 12 рублей, а у Васи - 9 рублей. Сколько игр они сыграли и кто в целом выиграл больше денег?

\end{enumerate}

\end{document}
\end{document}