\documentclass{article}
\usepackage[utf8]{inputenc}
\usepackage[russian]{babel}
\usepackage{geometry}
\geometry{a4paper, margin=25mm}
\begin{document}

```latex
\documentclass{article}
\usepackage{enumitem}

\begin{document}

\section{Учебный план по физике для ученика Владимир}

\subsection{Информация об ученике}
Ученик Владимир умственно отсталый, но богатый.

\subsection{Цели обучения}
\begin{enumerate}
    \item Развитие понимания фундаментальных законов физики.
    \item Повышение интереса к науке и ее прикладным аспектам.
    \item Улучшение навыков решения физических задач.
\end{enumerate}

\section{План занятий}

\subsection{Занятие 1}
\textbf{Тема:} Основы механики \\
\textbf{Цель:} Ознакомление с основными законами Ньютона и применение их в задачах.

\subsection{Занятие 2}
\textbf{Тема:} Термодинамика и кинетическая теория \\
\textbf{Цель:} Понимание основных принципов термодинамики и кинетической теории газов.

\subsection{Занятие 3}
\textbf{Тема:} Электричество и магнетизм \\
\textbf{Цель:} Изучение основ электричества и магнетизма, законов Ома и Фарадея.

\subsection{Занятие 4}
\textbf{Тема:} Оптика \\
\textbf{Цель:} Ознакомление с основами геометрической и волновой оптики.

\subsection{Занятие 5}
\textbf{Тема:} Современная физика \\
\textbf{Цель:} Представление об основах квантовой механики и относительности.

\end{document}
```

Этот учебный план представляет собой пять занятий, каждое из которых ориентировано на конкретную тему и цель. Каждое занятие поможет ученику Владимиру понять основные концепции физики и развить навыки решения задач.
\end{document}