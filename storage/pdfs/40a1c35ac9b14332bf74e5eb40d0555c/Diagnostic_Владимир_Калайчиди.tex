\documentclass{article}
\usepackage[utf8]{inputenc}
\usepackage[russian]{babel}
\usepackage{geometry}
\geometry{a4paper, margin=25mm}
\begin{document}

Тест по математике для ученика Владимира:

Часть 1: Арифметика

1. Решите уравнение: \(3x + 5 = 17\).
2. Вычислите \(4^2 \times 2 + 10\).
3. Найдите корень уравнения: \(x^2 - 9 = 0\).
4. Переведите число \(1011_2\) из двоичной системы счисления в десятичную систему.
5. Решите задачу: Если у Анны 3 яблока, а у Пети в 2 раза больше, сколько яблок у Пети?

Часть 2: Геометрия

6. Найдите площадь прямоугольника со сторонами 6 см и 9 см.
7. Найдите периметр квадрата со стороной 5 см.
8. Чему равен объем параллелепипеда, если его длина 4 см, ширина 3 см, высота 2 см?
9. Найдите значение угла, если сумма его дополнительного и смежного углов равна 140 градусов.
10. Решите задачу: Если радиус окружности равен 6 см, найдите длину окружности.

Часть 3: Теория чисел

11. Является ли число 57 простым?
12. Найдите НОД чисел 24 и 36.
13. Какой будет остаток от деления числа 125 на 7?
14. Найдите сумму всех чисел от 1 до 100.
15. Решите задачу: Если число делится на 3 и на 5, будет ли оно делиться на 15?

Часть 4: Алгебра

16. Решите систему уравнений: \(\begin{cases} 2x + y = 5 \\ x - y = 1 \end{cases}\).
17. Разложите на множители: \(x^2 - 4y^2\).
18. Найдите значение выражения: \(\frac{3x^2}{y} - 2y\) при \(x = 4\) и \(y = 2\).
19. Решите уравнение: \(x^2 - 6x + 9 = 0\).
20. Решите задачу: Если \(y = 2x + 1\) и \(x = 3\), найдите значение \(y\).

Часть 5: Сложные задачи

21. В классе 30 учеников, из них 20 изучают физику, 15 изучают математику, а 5 изучают оба предмета. Сколько учеников не изучают ни математику, ни физику?
22. Машина проехала 240 км за 4 часа. С какой скоростью она двигалась?
23. В треугольнике ABC угол A равен 60 градусов, сторона AB равна 8 см. Найдите длину стороны AC.
24. Найдите среднее арифметическое всех двузначных чисел.
25. У Владимира было 30 яблок. Он съел 2/3 от общего количества. Сколько яблок у него осталось?

Этот тест охватывает различные темы математики, проверяет знание основных понятий, навыки решения уравнений и задач, а также способность применять математические знания на практике.
\end{document}