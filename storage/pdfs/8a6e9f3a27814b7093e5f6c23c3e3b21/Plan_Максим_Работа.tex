\documentclass{article}
\usepackage[utf8]{inputenc}
\usepackage[russian]{babel}
\usepackage{geometry}
\geometry{a4paper, margin=25mm}
\begin{document}

```latex
\section*{Учебный план по предмету \textbf{Физика} для ученика Максим}

\textbf{Уровень:} Хороший парень, старательный

\textbf{Текущий этап:} Начинаем изучение раздела \textit{Термодинамика}

\vspace{0.5cm}

\begin{tabular}{|p{3cm}|p{7cm}|p{5cm}|}
\hline
\textbf{Занятие} & \textbf{Тема} & \textbf{Цели и задачи} \\
\hline
1 & Введение в термодинамику. Основные понятия и величины & Познакомиться с термодинамикой как разделом физики, понять систему и её параметры (температура, давление, объём, внутренняя энергия). Усвоить основные термины: термодинамическая система, состояние, процесс. \\
\hline
2 & Температура и термометрия & Изучить физическую сущность температуры, познакомиться с разными температурными шкалами (Цельсия, Кельвина). Понять принцип работы термометров и способы измерения температуры. \\
\hline
3 & Первое начало термодинамики & Освоить формулировку первого закона термодинамики. Понять взаимосвязь внутренней энергии, работы и теплоты. Решать простые задачи на вычисление работы и изменения внутренней энергии. \\
\hline
4 & Внутренняя энергия идеального газа & Изучить понятие внутренней энергии идеального газа. Связь внутренней энергии с температурой. Понять, что внутренняя энергия зависит только от температуры для идеального газа. \\
\hline
5 & Изопроцессы в термодинамике (изохорный, изобарный, изотермический, адиабатный) & Рассмотреть особенности каждого изопроцесса, научиться записывать уравнения состояния и рассчитывать изменения параметров. Решать задачи на изопроцессы. \\
\hline
6 & Работа газа при расширении и сжатии & Понять физический смысл работы газа. Научиться вычислять работу при различных термодинамических процессах. Интерпретировать графики в координатах P-V. \\
\hline
7 & Второе начало термодинамики. Энтропия & Ввести понятие второго начала термодинамики. Понять смысл энтропии и необратимости процессов. Рассмотреть примеры и формулировки второго закона. \\
\hline
8 & Циклы тепловых двигателей и КПД & Изучить термодинамические циклы (Цикл Карно и другие). Понять понятие коэффициента полезного действия (КПД). Рассчитать КПД простых тепловых машин. \\
\hline
9 & Практическое занятие: решение комплексных задач по термодинамике & Закрепить материал курса, решая задачи на различные аспекты термодинамики, включая применение первого и второго законов, расчёты работы и теплот. \\
\hline
10 & Итоговое повторение и систематизация знаний & Повторить ключевые понятия и формулы, ответить на вопросы, подготовиться к контрольной работе. \\
\hline
\end{tabular}

\vspace{0.5cm}

\textbf{Примечания:} Занятия планируются с учётом темпа Максим и его активности. Дополнительно рекомендуется использовать демонстрационные опыты и видео-материалы для лучшего понимания процессов.

```
Если нужно, могу составить расписание по дням или добавить список рекомендуемой литературы.
\end{document}