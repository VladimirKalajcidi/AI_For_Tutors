\documentclass{article}
\usepackage[utf8]{inputenc}
\usepackage[russian]{babel}
\usepackage{geometry}
\geometry{a4paper, margin=25mm}
\begin{document}

\documentclass{article}
\usepackage[utf8]{inputenc}
\usepackage{amsmath}

\begin{document}

\section*{Задание по математике для ученика Вадима}

\subsection*{Тема: Геометрия}

\begin{enumerate}
    \item Найдите площадь правильного шестиугольника со стороной 10 см.
    \item Рассмотрим треугольник со сторонами 7 см, 24 см и 25 см. Является ли он прямоугольным? Если да, то почему?
    \item Найдите угол между минутной и часовой стрелками в момент, когда время показывает 3:20.
    \item Рассмотрим окружность с радиусом 10 см. Найдите длину дуги, которую она подсветила, если угол между лучами, опирающимися на концы дуги, равен \( \frac{3\pi}{4} \) радиан.
    \item Постройте треугольник ABC, зная что сторона AB = 5 см, угол B = 60 градусов, и сторона BC = 7 см.
\end{enumerate}

Ученик Вадим, не забудь включить мозг и решить эти задачи! Успехов в изучении математики!

\end{document}
\end{document}