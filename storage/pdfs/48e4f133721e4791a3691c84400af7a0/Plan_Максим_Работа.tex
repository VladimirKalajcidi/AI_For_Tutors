\documentclass{article}
\usepackage[utf8]{inputenc}
\usepackage[russian]{babel}
\usepackage{geometry}
\geometry{a4paper, margin=25mm}
\begin{document}

Конечно, вот учебный план по предмету Физика для Максима:

\documentclass{article}
\usepackage{enumitem}

\begin{document}

\title{\textbf{Учебный план по Физике для Максима}}
\author{}
\date{}
\maketitle

\section*{Уровень: Хороший парень, старательный}

\section*{Тема: Термодинамика}

\begin{enumerate}[label=\textbf{\arabic*.}, leftmargin=*]
  \item \textbf{Введение в термодинамику}
    \begin{itemize}
        \item Цель: Ознакомление с основными понятиями и законами термодинамики.
        \item Занятие 1: Понятие термодинамики, термодинамические системы.
        \item Занятие 2: Первое начало термодинамики, внутренняя энергия системы.
    \end{itemize}
  
  \item \textbf{Тепловые процессы}
    \begin{itemize}
        \item Цель: Изучение различных тепловых процессов.
        \item Занятие 3: Адиабатический, изобарный и изохорный процессы.
        \item Занятие 4: Циклы тепловых машин, КПД.
    \end{itemize}
  
  \item \textbf{Второе начало термодинамики}
    \begin{itemize}
        \item Цель: Понимание второго начала термодинамики и его следствий.
        \item Занятие 5: Формулировка второго начала термодинамики, тепловые насосы и холодильники.
        \item Занятие 6: Энтропия, термодинамический потенциал.
    \end{itemize}
\end{enumerate}

\end{document} 

План включает введение в термодинамику, изучение тепловых процессов и второго начала термодинамики. Каждая тема разбита на несколько занятий с указанием целей и содержания. Надеюсь, этот план будет полезным для обучения Максима!
\end{document}