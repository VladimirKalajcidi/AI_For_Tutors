\documentclass{article}
\usepackage[utf8]{inputenc}
\usepackage[russian]{babel}
\usepackage{geometry}
\usepackage{amsmath}
\geometry{a4paper, margin=25mm}
\begin{document}
Диагностический тест по предмету «Физика» для ученика Максим

Цель: выявить уровень знаний по ключевым темам, а также определить пробелы и трудности в понимании материала.

Инструкция: внимательно прочитайте каждый вопрос и выберите или запишите ответ. В тесте используются разные типы заданий: выбор ответа, вычислительные задачи, вопросы на сопоставление и открытые вопросы.

---

### 1. Механика

**1.1. Теория и понятия**

1) Что такое сила?  
a) Величина, характеризующая скорость тела  
b) Векторная величина, характеризующая взаимодействие между телами  
c) Величина, показывающая массу тела  
d) Величина, равная произведению массы на скорость

2) Какая из следующих формул соответствует второму закону Ньютона?  
a) F = m / a  
b) F = m × a  
c) F = m + a  
d) F = m - a

**1.2. Вычислительная задача**

3) Тело массой 5 кг движется с ускорением 2 м/с². Найдите силу, действующую на тело.

---

### 2. Тепловые явления

**2.1. Теория**

4) Что происходит с внутренней энергией тела при нагревании?  
a) Уменьшается  
b) Не меняется  
c) Увеличивается  
d) Сначала уменьшается, затем увеличивается

5) Как называется процесс передачи теплоты через соприкосновение тел?  
a) Конвекция  
b) Излучение  
c) Теплопроводность  
d) Изоляция

**2.2. Ловушка**

6) Термометр показывает температуру в 0 °C. При этом молекулы воды:  
a) Совершают только колебательные движения  
b) Совершают только поступательные движения  
c) Совершают колебания, вращения и поступательные движения  
d) Не движутся

---

### 3. Электричество и магнетизм

**3.1. Теория**

7) Какой из законов описывает зависимость силы тока от напряжения и сопротивления?  
a) Закон Ома  
b) Закон Кулона  
c) Закон сохранения энергии  
d) Закон Архимеда

8) Как изменится сопротивление проводника при увеличении его длины в 3 раза и уменьшении площади поперечного сечения в 2 раза?  
a) Увеличится в 6 раз  
b) Увеличится в 1.5 раза  
c) Уменьшится в 6 раз  
d) Останется без изменений

**3.2. Вычислительная задача**

9) В цепи сопротивлением 10 Ом течёт ток 2 А. Найдите напряжение на цепи.

---

### 4. Оптика

**4.1. Теория**

10) Каким изображением обладает предмет, расположенный перед выпуклой линзой дальше её фокуса?  
a) Мнимым, увеличенным  
b) Действительным, уменьшенным  
c) Действительным, увеличенным  
d) Мнимым, уменьшенным

11) Что происходит с лучом света при прохождении из воздуха в воду?  
a) Он ускоряется  
b) Он изменяет направление  
c) Он отражается полностью  
d) Он не изменяет скорость

**4.2. Ловушка**

12) Луч света падает на границу двух сред под углом 90°. Что произойдет?  
a) Луч преломится  
b) Луч полностью отразится  
c) Луч продолжит движение без отклонения  
d) Луч поглотится

---

### 5. Астрономия и космология

**5.1. Теория**

13) Что такое световой год?  
a) Расстояние, которое свет проходит за 1 секунду  
b) Время, за которое свет проходит 1 км  
c) Время, за которое свет проходит 1 год  
d) Расстояние, которое свет проходит за 1 год

14) Какое из утверждений неверно?  
a) Земля вращается вокруг своей оси за 24 часа  
b) Луна — звезда  
c) Солнце — звезда  
d) Планеты вращаются вокруг Солнца

---

### 6. Заключительный блок: открытые вопросы

15) Объясните, почему тело, брошенное вертикально вверх, на вершине траектории имеет скорость равную нулю, но ускорение не равно нулю.

16) Опишите, как изменится сила тока в цепи, если напряжение увеличить вдвое, а сопротивление оставить постоянным.

---

### Критерии оценки и анализ результатов

- Верные ответы на теоретические вопросы показывают понимание базовых понятий.  
- Решение вычислительных задач демонстрирует навыки применения формул и расчетов.  
- Вопросы-ловушки выявляют поверхностное понимание и распространённые ошибки.  
- Открытые вопросы позволяют оценить умение формулировать и объяснять физические явления.

---

Если желаете, могу подготовить ключи с подробным разбором ответов.
\end{document}