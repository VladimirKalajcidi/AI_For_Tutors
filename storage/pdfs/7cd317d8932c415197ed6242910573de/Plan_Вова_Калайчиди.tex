\documentclass{article}
\usepackage[utf8]{inputenc}
\usepackage[russian]{babel}
\usepackage{geometry}
\geometry{a4paper, margin=25mm}
\begin{document}

```latex
\documentclass{article}

\usepackage{enumitem}

\begin{document}

\section*{Учебный план по математике для ученика Вовы}

\subsection*{Исходные данные об ученике}
Умственно отсталый.

\subsection*{Цели обучения}
\begin{itemize}
    \item Развитие базовых навыков математики.
    \item Повышение уровня понимания математических концепций.
    \item Улучшение способностей к решению математических задач.
\end{itemize}

\section{Темы и задачи}

\subsection{Занятие 1: Введение в числовые понятия}
\begin{itemize}
    \item Цель: Освоение основных числовых понятий.
    \item Темы:
    \begin{itemize}
        \item Числа от 1 до 10.
        \item Сложение и вычитание чисел до 10.
    \end{itemize}
\end{itemize}

\subsection{Занятие 2: Работа с числами}
\begin{itemize}
    \item Цель: Улучшение навыков работы с числами.
    \item Темы:
    \begin{itemize}
        \item Умножение и деление чисел до 10.
        \item Решение простых математических примеров.
    \end{itemize}
\end{itemize}

\subsection{Занятие 3: Геометрические фигуры}
\begin{itemize}
    \item Цель: Знакомство с основными геометрическими фигурами.
    \item Темы:
    \begin{itemize}
        \item Круг, квадрат, треугольник, прямоугольник.
        \item Различие между фигурами.
    \end{itemize}
\end{itemize}

\subsection{Занятие 4: Основы счёта}
\begin{itemize}
    \item Цель: Развитие навыков счёта.
    \item Темы:
    \begin{itemize}
        \item Сравнение чисел.
        \item Порядковые числа.
    \end{itemize}
\end{itemize}

\subsection{Занятие 5: Решение задач}
\begin{itemize}
    \item Цель: Научиться решать простейшие математические задачи.
    \item Темы:
    \begin{itemize}
        \item Понимание условий задачи.
        \item Практическое решение задач.
    \end{itemize}
\end{itemize}

\end{document}
```

Этот учебный план содержит пять занятий по различным темам математики, каждое из которых имеет свою цель и набор тем. Каждое занятие направлено на развитие определенных навыков и знаний у ученика.
\end{document}