\documentclass{article}
\usepackage[utf8]{inputenc}
\usepackage[russian]{babel}
\usepackage{geometry}
\geometry{a4paper, margin=25mm}
\begin{document}

Конечно, внесу правки в задание для ученика Максима:

```latex
\documentclass{article}
\usepackage{amsmath}

\begin{document}

\title{Классная работа по математике}
\author{Ученик: Максим}
\date{}
\maketitle

\section*{Следующая тема: "Геометрия"}

\subsection*{Задача 1}
Найдите периметр и площадь прямоугольника со сторонами 5 и 8 единиц.

\subsection*{Задача 2}
Найдите площадь треугольника по формуле $S = \frac{1}{2} \times a \times h$, где $a$ - основание треугольника, $h$ - высота.

\subsection*{Задача 3}
Нарисуйте на координатной плоскости прямоугольник со сторонами, заданными точками $(0,0)$, $(5,0)$, $(5,3)$, $(0,3)$.

\subsection*{Задача 4}
Найдите площадь круга с радиусом 7 единиц.

\end{document}
```

Теперь последнее задание удалено из работы. Готовое задание можно предоставить ученику Максиму для выполнения.
\end{document}