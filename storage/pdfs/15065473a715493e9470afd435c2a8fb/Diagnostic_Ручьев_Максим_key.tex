\documentclass{article}
\usepackage[utf8]{inputenc}
\usepackage[russian]{babel}
\usepackage{geometry}
\geometry{a4paper, margin=25mm}
\begin{document}

Для удобства я разбил ключ ответов на каждый вопрос:

```latex
\documentclass{article}
\usepackage{amsmath}

\begin{document}

\section*{Ключ ответов к диагностическому тесту по математике для ученика Максима}

\textbf{Часть 1: Основные арифметические операции}

1. 
\begin{enumerate}
    \item a) \(5 + 3 \times 2 = 5 + 6 = 11\)
    \item b) \(12 - (4 \div 2) = 12 - 2 = 10\)
    \item c) \((6 - 3) \times 4 = 3 \times 4 = 12\)
\end{enumerate}

2. 
\begin{enumerate}
    \item a) \(3^2 + 4 \times 2 = 9 + 8 = 17\)
    \item b) \((8 - 3) \div 5 = 5 \div 5 = 1\)
    \item c) \(\sqrt{16} - 2 \times 3 = 4 - 6 = -2\)
\end{enumerate}

3. 
\begin{enumerate}
    \item a) \( \frac{3}{5} = 0.6 \)
    \item b) \( \frac{7}{8} = 0.875 \)
    \item c) \( \frac{2}{3} = 0.\overline{6} \)
\end{enumerate}

\textbf{Часть 2: Геометрия}

4. 
\begin{itemize}
    \item Периметр прямоугольника: \(2 \times (5 + 8) = 26\) единиц
    \item Площадь прямоугольника: \(5 \times 8 = 40\) квадратных единиц
\end{itemize}

5. 
\begin{itemize}
    \item Площадь треугольника: \(S = \frac{1}{2} \times 5 \times h = 2.5h\), где h - высота треугольника
\end{itemize}

\textbf{Часть 3: Уравнения и неравенства}

6. 
\begin{enumerate}
    \item a) \(2x + 5 = 11 \Rightarrow 2x = 6 \Rightarrow x = 3\)
    \item b) \(3(x - 4) = 15 \Rightarrow 3x - 12 = 15 \Rightarrow 3x = 27 \Rightarrow x = 9\)
\end{enumerate}

7. 
\begin{enumerate}
    \item a) \(2x + 7 < 15 \Rightarrow 2x < 8 \Rightarrow x < 4\)
    \item b) \(-3(x - 2) \geq 6 \Rightarrow -3x + 6 \geq 6 \Rightarrow -3x \geq 0 \Rightarrow x \leq 0\)
\end{enumerate}

\textbf{Часть 4: Сложные задачи}

8. Поезд проедет 300 км за \( \frac{300}{60} = 5 \) часов.

9. Цена 2.5 кг яблок: \(25 \times 2.5 = 62.5\) рублей.

\textbf{Часть 5: Логика и анализ}

10. Самый молодой из них - Иван.

\end{document}
```
\end{document}