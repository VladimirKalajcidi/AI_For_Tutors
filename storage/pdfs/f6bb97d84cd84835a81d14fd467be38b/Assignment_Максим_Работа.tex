\documentclass{article}
\usepackage[utf8]{inputenc}
\usepackage[russian]{babel}
\usepackage{geometry}
\geometry{a4paper, margin=25mm}
\begin{document}

```latex
\documentclass[a4paper,12pt]{article}
\usepackage{amsmath}
\usepackage{amssymb}

\begin{document}

\title{Задание по теме: Термодинамика}
\author{Для ученика Максим, уровень: Хороший парень, старательный}
\date{}
\maketitle

\vspace{0.5cm}

\textbf{Инструкция:} Решите следующие нестандартные задачи по теме "Термодинамика". Внимательно проанализируйте условия, используйте основные законы термодинамики и при необходимости сделайте обоснованные допущения.

\vspace{0.5cm}

\begin{enumerate}

\item \textbf{Адиабатический процесс с изменяющимся объёмом.} \\
Идеальный газ в цилиндре с поршнем изначально находится в состоянии с объёмом $V_1 = 2\,\mathrm{л}$ и температурой $T_1 = 300\,\mathrm{K}$. Газ расширяется адиабатически так, что объём увеличивается в 4 раза. Найдите конечную температуру газа $T_2$, если показатель адиабаты $\gamma = 1.4$.

\vspace{0.5cm}

\item \textbf{Цикл с изохорическим и изобарическим процессами.} \\
Газ сначала нагревается изохорически от $T_1 = 290\,\mathrm{K}$ до $T_2 = 350\,\mathrm{K}$, затем расширяется изобарически, увеличивая свой объём в 2 раза. Определите изменение внутренней энергии и работу газа за оба процесса, если количество вещества равно $n=1\,\mathrm{моль}$, и газ идеальный одноатомный ($C_V = \frac{3}{2}R$).

\vspace{0.5cm}

\item \textbf{Теплообмен между двумя телами.} \\
Два тела одинаковой массы: одно из алюминия, другое из меди, имеют начальные температуры $80^\circ C$ и $20^\circ C$ соответственно. После теплового контакта установилась общая температура $T$. Определите $T$, если теплообмен происходит без потерь. Удельные теплоёмкости: алюминий $c_{Al} = 900\,\frac{\mathrm{Дж}}{\mathrm{кг}\cdot^\circ C}$, меди $c_{Cu} = 385\,\frac{\mathrm{Дж}}{\mathrm{кг}\cdot^\circ C}$.

\vspace{0.5cm}

\item \textbf{КПД теплового двигателя.} \\
Тепловой двигатель работает между двумя резервуарами с температурами $T_{гор} = 600\,\mathrm{K}$ и $T_{хол} = 300\,\mathrm{K}$. За один цикл двигатель совершает работу $W = 150\,\mathrm{Дж}$ и получает тепло $Q_{гор}$ от горячего резервуара. Найдите количество теплоты $Q_{хол}$, отданное холодному резервуару, и КПД двигателя.

\vspace{0.5cm}

\item \textbf{Изменение энтропии при смешении двух газов.} \\
Два одинаковых объёма идеального газа при одинаковой температуре $T = 300\,\mathrm{K}$ и давлении $P = 1\,\mathrm{атм}$ находятся в соседних сосудах, разделённых перегородкой. Перегородка удаляется, и газы смешиваются без теплообмена с окружающей средой и без совершения работы. Определите изменение энтропии системы, если газы различаются по составу. Количество вещества в каждом сосуде $n=1\,\mathrm{моль}$.

\end{enumerate}

\vspace{0.5cm}

\textbf{Желаю успехов!}

\end{document}
```
\end{document}