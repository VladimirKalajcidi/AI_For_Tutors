\documentclass{article}
\usepackage[utf8]{inputenc}
\usepackage[russian]{babel}
\usepackage{geometry}
\geometry{a4paper, margin=25mm}
\begin{document}

Диагностический тест по физике для ученика Максим:

Часть 1. Механика

1. Чем отличается скорость от ускорения?
а) Скорость - векторная величина, ускорение - скалярная
б) Скорость - скалярная величина, ускорение - векторная
в) Скорость и ускорение обе векторные величины
г) Скорость и ускорение обе скалярные величины

2. Чему равно ускорение свободного падения на Земле?
а) 9,8 м/с²
б) 10 м/с²
в) 5 м/с²
г) 12 м/с²

3. Какая формула используется для расчета пути равноускоренного движения?
а) s = v₀t + (at²)/2
б) s = v₀t + at
в) s = v₀t + a
г) s = v₀ + at

Часть 2. Термодинамика

4. Что такое адиабатический процесс?
а) Процесс, при котором система находится в тепловом равновесии с окружающей средой
б) Процесс, при котором в системе нет теплообмена с окружающей средой
в) Процесс, при котором в системе происходит равновесное расширение
г) Процесс, при котором в системе происходит равновесное сжатие

5. Какой закон формулирует закон сохранения энергии?
а) Закон Ньютона
б) Закон Ампера
в) Закон Гесса
г) Закон Грейса

6. Что такое теплоемкость?
а) Количество теплоты, необходимое для нагревания вещества на 1 градус Цельсия
б) Способность вещества поглощать теплоту
в) Количество теплоты, необходимое для изменения температуры вещества на 1 К
г) Способность вещества отдавать теплоту

Часть 3. Электричество и магнетизм

7. Какая формула позволяет вычислить силу тока в цепи?
а) I = V/R
б) I = VR
в) I = R/V
г) I = V + R

8. Что такое электрическое поле?
а) Область пространства, в которой действует электрическая сила
б) Сила, с которой электрический заряд действует на другой заряд
в) Ток, проходящий через проводник
г) Разность потенциалов между двумя точками

9. Как называется явление изменения направления тока в проводнике при изменении магнитного поля?
а) Электромагнитная индукция
б) Электростатический эффект
в) Электрический ток
г) Электрическая емкость

Часть 4. Оптика

10. Что такое полное внутреннее отражение света?
а) Явление, при котором свет не проникает из оптически более плотной среды в менее плотную
б) Явление, при котором свет не проникает из оптически менее плотной среды в более плотную
в) Явление, при котором свет поглощается средой
г) Явление, при котором свет рассеивается

11. Какая формула определяет фокусное расстояние линзы?
а) 1/f = 1/d₀ + 1/dᵢ
б) f = d₀ + dᵢ
в) f = d₀/dᵢ
г) f = d₀ - dᵢ

12. Что такое дисперсия света?
а) Распространение света в среде
б) Разложение света на спектральные составляющие
в) Изгиб света при прохождении через оптическую систему
г) Отражение света от поверхности

Часть 5. Атомная и ядерная физика

13. Какой элементарной частицей является протон?
а) Бозон
б) Лептон
в) Фермион
г) Кварк

14. Что такое радиоактивность?
а) Способность вещества испускать радиоактивное излучение
б) Способность вещества поглощать радиоактивное излучение
в) Способность вещества изменять свою массу под воздействием радиации
г) Способность вещества изменять свою структуру под воздействием радиации

15. Как называется процесс деления ядра атома на более легкие ядра?
а) Ядерный синтез
б) Ядерный распад
в) Ядерная реакция
г) Ядерное деление

Этот тест поможет выявить пробелы в знаниях Максима по различным темам физики и позволит определить уровень его подготовки.
\end{document}