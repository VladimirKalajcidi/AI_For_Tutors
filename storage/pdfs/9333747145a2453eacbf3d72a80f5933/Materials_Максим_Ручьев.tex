\documentclass{article}
\usepackage[utf8]{inputenc}
\usepackage[russian]{babel}
\usepackage{geometry}
\geometry{a4paper, margin=25mm}
\begin{document}

Конечно, вот обучающие материалы по следующей теме из учебного плана по предмету Математика для ученика Максима:

\documentclass{article}
\usepackage{amsmath}

\begin{document}

\title{Учебные материалы по Математике}
\author{Ученик: Максим}
\date{}
\maketitle

\section*{Следующая тема: "Подготовка к ЕГЭ"}

\subsection*{Теория}
Подготовка к экзамену ЕГЭ включает в себя решение разнообразных задач по математике. Важно уметь работать с арифметическими прогрессиями, уравнениями, геометрическими фигурами и т.д. Давай рассмотрим несколько ключевых понятий:

\begin{itemize}
    \item **Арифметическая прогрессия** - последовательность чисел, в которой каждый следующий член получается прибавлением к предыдущему одного и того же постоянного числа.
    
    \item **Площадь круга** вычисляется по формуле $S = \pi r^2$, где $r$ - радиус круга.
    
    \item **Решение уравнений** - найти значение переменной, удовлетворяющее уравнению.
    
    \item **Разложение на множители** - представление выражения в виде произведения множителей.
\end{itemize}

\subsection*{Примеры}

1. **Арифметическая прогрессия**: Найдем сумму первых 5 членов арифметической прогрессии с первым членом 2 и разностью 3.

2. **Площадь круга**: Вычислим площадь круга с радиусом 10 единиц.

3. **Решение уравнения**: Решим уравнение $2x + 3 = 11$.

4. **Разложение на множители**: Разложим на множители выражение $x^2 - 4$.

\subsection*{Упражнения}

1. Решите уравнение: $3(x - 4) = 15$.

2. Найдите сумму первых 8 членов арифметической прогрессии со значением первого члена 4 и разностью 2.

3. Вычислите площадь треугольника с основанием 6 и высотой 8.

4. Разложите на множители выражение $x^2 - 5x + 6$.

\end{document}

Эти материалы помогут Максиму понять и закрепить тему "Подготовка к ЕГЭ" по математике. Успехов в изучении!
\end{document}