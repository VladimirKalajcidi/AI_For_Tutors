\documentclass{article}
\usepackage[utf8]{inputenc}
\usepackage[russian]{babel}
\usepackage{geometry}
\geometry{a4paper, margin=25mm}
\begin{document}

Конечно, вот обучающие материалы по следующей теме из учебного плана по предмету Математика для ученика Максима:

\documentclass{article}
\usepackage{amsmath}

\begin{document}

\title{Следующая тема: "Углы и их измерение"}
\author{Ученик: Максим}
\date{}
\maketitle

\section{Теория}
Угол - это часть плоскости, ограниченная двумя полулиниями, исходящими из общего начала. Угол измеряется в градусах.

\subsection{Виды углов}
\begin{itemize}
    \item \textbf{Острый угол:} угол, меньший 90 градусов.
    \item \textbf{Прямой угол:} угол, равный 90 градусов.
    \item \textbf{Тупой угол:} угол, больший 90 градусов, но меньший 180 градусов.
\end{itemize}

\section{Примеры упражнений}

\subsection{Упражнение 1}
Найдите меру угла, если его дополнительный угол равен 40 градусов.

\subsection{Упражнение 2}
Рассмотрим треугольник со следующими углами: $30^\circ$, $60^\circ$, $90^\circ$. Найдите третий угол.

\subsection{Упражнение 3}
Даны углы: $x$, $2x$, $3x$. Найдите их сумму, если $x = 20^\circ$.

\section{Заключение}
Знание углов и их измерение важно для решения геометрических задач. Постоянная практика поможет вам лучше понять эту тему.

\end{document}

Надеюсь, эти материалы будут полезными для изучения темы "Углы и их измерение". Удачи в изучении математики!
\end{document}