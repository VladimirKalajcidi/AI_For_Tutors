\documentclass{article}
\usepackage[utf8]{inputenc}
\usepackage[russian]{babel}
\usepackage{geometry}
\geometry{a4paper, margin=25mm}
\begin{document}

Диагностический тест по математике для ученика Максима:

Часть 1: Основные арифметические операции
1. Решите следующие уравнения:
   a) \(5 + 3 \times 2\)
   b) \(12 - (4 \div 2)\)
   c) \((6 - 3) \times 4\)

2. Вычислите:
   a) \(3^2 + 4 \times 2\)
   b) \((8 - 3) \div 5\)
   c) \(\sqrt{16} - 2 \times 3\)

3. Переведите в десятичную дробь:
   a) \( \frac{3}{5} \)
   b) \( \frac{7}{8} \)
   c) \( \frac{2}{3} \)

Часть 2: Геометрия
4. Найдите периметр и площадь прямоугольника со сторонами 5 и 8 единиц.
5. Найдите площадь треугольника по формуле \(S = \frac{1}{2} \times a \times h\), где a - основание треугольника, h - высота.

Часть 3: Уравнения и неравенства
6. Решите уравнения:
   a) \(2x + 5 = 11\)
   b) \(3(x - 4) = 15\)

7. Решите неравенства:
   a) \(2x + 7 < 15\)
   b) \(-3(x - 2) \geq 6\)

Часть 4: Сложные задачи
8. Поезд движется со скоростью 60 км/ч. Сколько времени ему потребуется, чтобы проехать 300 км?
9. В магазине продавались яблоки по цене 25 рублей за килограмм. Сколько стоит 2.5 кг яблок?

Часть 5: Логика и анализ
10. Если Анна старше, чем Мария, а Мария старше, чем Иван, то кто из них самый молодой?

Пожалуйста, пройдите этот тест и затем мы сможем обсудить результаты и выявить пробелы в знаниях.
\end{document}