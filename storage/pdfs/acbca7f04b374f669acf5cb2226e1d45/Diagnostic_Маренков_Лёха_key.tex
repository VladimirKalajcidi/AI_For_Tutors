\documentclass{article}
\usepackage[utf8]{inputenc}
\usepackage[russian]{babel}
\usepackage{geometry}
\geometry{a4paper, margin=25mm}
\begin{document}

```latex
\documentclass{article}
\usepackage{amsmath}
\usepackage{amsfonts}

\begin{document}

\title{Диагностический тест по Математике}
\author{Лёха}
\date{\today}

\maketitle

\section{Простые арифметические действия}

\begin{enumerate}
    \item[a)] $25 + 14 = 39$
    \item[b)] $36 - 19 = 17$
    \item[c)] $7 \times 8 = 56$
    \item[d)] $45 / 5 = 9$
\end{enumerate}

\section{Решение уравнений}

\begin{enumerate}
    \item[a)] $3x + 5 = 14$ \\
    $3x = 14 - 5$ \\
    $3x = 9$ \\
    $x = 3$
    
    \item[b)] $2y - 10 = 16$ \\
    $2y = 16 + 10$ \\
    $2y = 26$ \\
    $y = 13$
\end{enumerate}

\section{Геометрия}

\begin{enumerate}
    \item[a)] Периметр прямоугольника со сторонами 6 см и 10 см равен $2 \times (6 + 10) = 32$ см.
    
    \item[b)] Площадь круга с радиусом 5 см равна $\pi \times 5^2 = 25\pi$ кв. см.
\end{enumerate}

\section{Дроби и проценты}

\begin{enumerate}
    \item[a)] $\frac{3}{5} = 0.6 = 60\%$
    
    \item[b)] $25\%$ от числа 80 равно $0.25 \times 80 = 20$
\end{enumerate}

\section{Сложные задачи}

\begin{enumerate}
    \item[a)] На тренировку пришло 36 человек. Если $\frac{1}{4}$ из них были девочки, то количество мальчиков равно $36 - \frac{1}{4} \times 36 = 27$
    
    \item[b)] Тетя Марина потратила $3 \times 60 + 2 \times 40 = 180 + 80 = 260$ рублей
\end{enumerate}

\section{Логика и математические рассуждения}

\begin{enumerate}
    \item[a)] При $a = 5$ и $b = 3$, результат выражения $2a + 3b = 2 \times 5 + 3 \times 3 = 10 + 9 = 19$
    
    \item[b)] Из того, что все кошки любят молоко, следует, что Мурка как кошка тоже любит молоко (это необходимое, но не достаточное условие).
\end{enumerate}

\section{«Ловушки»}

\begin{enumerate}
    \item[a)] $0! = 1$
    
    \item[b)] $\sqrt{-9}$ не имеет действительного значения, так как извлечение корня из отрицательного числа не определено в действительных числах.
\end{enumerate}

\section{Бонусный вопрос}

\begin{enumerate}
    \item[a)] Правило упрощения выражения $3x + 2x$ — это закон коммутативности умножения, который позволяет объединить одинаковые переменные: $3x + 2x = 5x$
\end{enumerate}

\end{document}
```
\end{document}