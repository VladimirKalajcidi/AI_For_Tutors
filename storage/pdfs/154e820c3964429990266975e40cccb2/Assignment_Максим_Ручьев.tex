\documentclass{article}
\usepackage[utf8]{inputenc}
\usepackage[russian]{babel}
\usepackage{geometry}
\geometry{a4paper, margin=25mm}
\begin{document}

Конечно, вот задание по следующей теме из учебного плана по предмету Математика для ученика Максима:

\documentclass{article}
\usepackage{amsmath}

\begin{document}

\title{Домашнее задание по Математике}
\author{Ученик: Максим}
\date{}
\maketitle

\section*{Следующая тема: "Нестандартные задачи"}

\subsection*{Задача 1}
Найдите сумму первых 10 членов арифметической прогрессии, если первый член равен 3, а разность равна 4.

\subsection*{Задача 2}
Решите задачу: Что больше, площадь круга с радиусом 5 или площадь квадрата со стороной 6? (Используйте $\pi \approx 3.14$)

\subsection*{Задача 3}
Найдите корень уравнения: $x^3 - 6x^2 + 11x - 6 = 0$.

\subsection*{Задача 4}
Разложите на множители выражение: $x^2 + 5x + 6$.

\subsection*{Задача 5}
Известно, что произведение двух чисел равно 24, а их сумма равна 14. Найдите эти числа.

\end{document}

Такое задание содержит нестандартные задачи, которые помогут ученику Максиму развить аналитическое мышление и применить полученные знания на практике. Уровень сложности соответствует его текущим знаниям. Желаю успехов в выполнении заданий!
\end{document}