\documentclass{article}
\usepackage[utf8]{inputenc}
\usepackage[russian]{babel}
\usepackage{geometry}
\geometry{a4paper, margin=25mm}
\begin{document}

Конечно, вот дополнительные учебные материалы по математике для ученика Максима:

\begin{itemize}
    \item Статья "Основы арифметики":
    \begin{itemize}
        \item Ссылка: [Основы арифметики](ссылка)
        \item Описание: Статья о основных арифметических операциях, включая сложение, вычитание, умножение и деление.
    \end{itemize}
    
    \item Книга "Геометрия для начинающих":
    \begin{itemize}
        \item Ссылка: [Геометрия для начинающих](ссылка)
        \item Описание: Книга содержит базовые понятия по геометрии, такие как периметр, площадь, формулы для различных фигур.
    \end{itemize}
    
    \item Ресурс "Уравнения и неравенства":
    \begin{itemize}
        \item Ссылка: [Уравнения и неравенства](ссылка)
        \item Описание: Сборник задач и упражнений по решению уравнений и неравенств.
    \end{itemize}
    
    \item Статья "Логика и анализ":
    \begin{itemize}
        \item Ссылка: [Логика и анализ](ссылка)
        \item Описание: Статья о логических задачах и аналитическом мышлении, помогающая развить навыки рассуждения.
    \end{itemize}
\end{itemize}

Эти материалы помогут ученику Максиму улучшить свои знания по математике на уровне учебы в средней школе.
\end{document}