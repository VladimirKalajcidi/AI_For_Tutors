\documentclass{article}
\usepackage[utf8]{inputenc}
\usepackage[russian]{babel}
\usepackage{geometry}
\geometry{a4paper, margin=25mm}
\begin{document}

```latex
\documentclass{article}
\usepackage[utf8]{inputenc}

\begin{document}

\section*{Учебный план по математике для Вовы}

\subsection*{Ученик: Вова}

\subsection*{Цель: Помочь Вове освоить математику в доступной форме}

\section{Темы и задачи}

\subsection{Занятие 1}

\subsubsection*{Тема: Основы арифметики}
\begin{itemize}
    \item Введение в понятия числа, сложение и вычитание
    \item Цель: Освоение основных арифметических операций
\end{itemize}

\subsection{Занятие 2}

\subsubsection*{Тема: Умножение и деление}
\begin{itemize}
    \item Практические упражнения по умножению и делению
    \item Цель: Понимание и умение применять умножение и деление
\end{itemize}

\subsection{Занятие 3}

\subsubsection*{Тема: Геометрия}
\begin{itemize}
    \item Основы геометрических фигур и их свойств
    \item Цель: Освоение основ геометрии
\end{itemize}

\subsection{Занятие 4}

\subsubsection*{Тема: Дроби и проценты}
\begin{itemize}
    \item Задачи на сложение, вычитание дробей, расчет процентов
    \item Цель: Понимание и умение работать с дробями и процентами
\end{itemize}

\subsection{Занятие 5}

\subsubsection*{Тема: Уравнения и неравенства}
\begin{itemize}
    \item Решение уравнений и неравенств простого уровня
    \item Цель: Освоение базовых методов решения уравнений и неравенств
\end{itemize}

\end{document}
```
\end{document}