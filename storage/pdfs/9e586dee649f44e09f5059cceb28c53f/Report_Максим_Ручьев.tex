\documentclass{article}
\usepackage[utf8]{inputenc}
\usepackage[russian]{babel}
\usepackage{geometry}
\geometry{a4paper, margin=25mm}
\begin{document}

Конечно, вот структурированный отчет об успеваемости ученика Максима по предмету Математика:

---

**Отчет об успеваемости ученика Максима**

**Школа:** СУНЦ НГУ

---

**Диагностический тест**

Диагностический тест по математике для ученика Максима:

**Часть 1: Основные арифметические операции**
1. Решите следующие уравнения:
   - a) \(5 + 3 \times 2\)
   - b) \(12 - (4 \div 2)\)
   - c) \((6 - 3) \times 4\)

2. Вычислите:
   - a) \(3^2 + 4 \times 2\)
   - b) \((8 - 3) \div 5\)
   - c) \(\sqrt{16} - 2 \times 3\)

3. Переведите в десятичную дробь:
   - a) \( \frac{3}{5} \)
   - b) \( \frac{7}{8} \)
   - c) \( \frac{2}{3} \)

**Часть 2: Геометрия**
4. Найдите периметр и площадь прямоугольника со сторонами 5 и 8 единиц.
5. Найдите площадь треугольника по формуле \(S = \frac{1}{2} \times a \times h\), где a - основание треугольника, h - высота.

**Часть 3: Уравнения и неравенства**
6. Решите уравнения:
   - a) \(2x + 5 = 11\)
   - b) \(3(x - 4) = 15\)

7. Решите неравенства:
   - a) \(2x + 7 < 15\)
   - b) \(-3(x - 2) \geq 6\)

**Часть 4: Сложные задачи**
8. Поезд движется со скоростью 60 км/ч. Сколько времени ему потребуется, чтобы проехать 300 км?
9. В магазине продавались яблоки по цене 25 рублей за килограмм. Сколько стоит 2.5 кг яблок?

**Часть 5: Логика и анализ**
10. Если Анна старше, чем Мария, а Мария старше, чем Иван, то кто из них самый молодой?

---

**Домашнее задание**

\[
\text{Домашнее задание по математике для ученика Максима}
\]

\[
\text{Уровень ученика: Учиться в СУНЦ НГУ}
\]

**Часть 1: Основные арифметические операции**

1. Решите следующие уравнения:
   - a) \(5 + 3 \times 2\)
   - b) \(12 - (4 \div 2)\)
   - c) \((6 - 3) \times 4\)

2. Вычислите:
   - a) \(3^2 + 4 \times 2\)
   - b) \((8 - 3) \div 5\)
   - c) \(\sqrt{16} - 2 \times 3\)

3. Переведите в десятичную дробь:
   - a) \( \frac{3}{5} \)
   - b) \( \frac{7}{8} \)
   - c) \( \frac{2}{3} \)

**Часть 2: Геометрия**

4. Найдите периметр и площадь прямоугольника со сторонами 5 и 8 единиц.
5. Найдите площадь треугольника по формуле \(S = \frac{1}{2} \times a \times h\), где a - основание треугольника, h - высота.

6. Нарисуйте на координатной плоскости прямоугольник со сторонами, заданными точками \((0,0)\), \((5,0)\), \((5,3)\), \((0,3)\).

**Часть 3: Уравнения и неравенства**

7. Решите уравнения:
   - a) \(2x + 5 = 11\)
   - b) \(3(x - 4) = 15\)

8. Решите неравенства:
   - a) \(2x + 7 < 15\)
   - b) \(-3(x - 2) \geq 6\)

**Часть 4: Сложные задачи**

9. Поезд движется со скоростью 60 км/ч. Сколько времени ему потребуется, чтобы проехать 300 км?
10. В магазине продавались яблоки по цене 25 рублей за килограмм. Сколько стоит 2.5 кг яблок?

**Часть 5: Логика и анализ**

11. Если Анна старше, чем Мария, а Мария старше, чем Иван, то кто из них самый молодой?
12. Докажите или опровергните утверждение: "Если число делится на 3, то оно обязательно делится на 9".

---

Данное домашнее задание содержит разнообразные задачи по математике для ученика Максима. Уровень сложности соответствует его текущим знаниям и не включает дифференцирование и интегралы. При необходимости, буду рад помочь с решением задач или дополнительными пояснениями.

---

Также, внесены исправления в домашнее задание и учебный план, чтобы соответствовать требованиям. Ученику Максиму желаю успешного выполнения заданий и развития математических навыков!
\end{document}