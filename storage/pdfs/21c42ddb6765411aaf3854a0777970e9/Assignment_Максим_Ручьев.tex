\documentclass{article}
\usepackage[utf8]{inputenc}
\usepackage[russian]{babel}
\usepackage{geometry}
\geometry{a4paper, margin=25mm}
\begin{document}

Конечно, вот задание по следующей теме из учебного плана по математике для ученика Максима:

\documentclass{article}
\usepackage{amsmath}

\begin{document}

\title{Задание по математике}
\author{Ученик: Максим}
\date{}
\maketitle

\section*{Тема: Нестандартные математические задачи}

\begin{enumerate}
    \item Пусть имеется квадратный лист бумаги со стороной 10 см. Каким образом его можно разрезать на наименьшее количество прямоугольных частей одинаковой площади?
    
    \item Рассмотрим последовательность чисел: 1, 2, 3, 5, 8, 13, 21, ... Какое число следует после 21 и как оно получается?

    \item Найдите значение выражения:
    \[
    \frac{1}{1 + \frac{1}{1 + \frac{1}{1 + \frac{1}{2}}}}
    \]

    \item В семье у Максима есть три сестры, каждая из которых имеет двух братьев. Сколько членов в семье Максима?

    \item Рассмотрим арифметическую прогрессию: 3, 9, 15, 21, ... Какова сумма первых 10 членов этой прогрессии?
\end{enumerate}

\end{document}

Данное задание содержит нестандартные математические задачи, которые помогут ученику Максиму развить логическое мышление и креативный подход к решению задач. Уровень сложности соответствует его текущим знаниям и способностям. При необходимости, готов помочь с объяснением и подсказками.
\end{document}