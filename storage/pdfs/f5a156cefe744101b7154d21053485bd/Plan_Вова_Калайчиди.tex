\documentclass{article}
\usepackage[utf8]{inputenc}
\usepackage[russian]{babel}
\usepackage{geometry}
\geometry{a4paper, margin=25mm}
\begin{document}

```latex
\documentclass{article}
\usepackage[utf8]{inputenc}

\title{Учебный план по математике для ученика Вовы}
\author{}
\date{}

\begin{document}

\maketitle

\section{Информация об ученике}
Ученик: Вова\\
Уровень: Умственно отсталый

\section{Учебный план}

\subsection{Занятие 1}
\subsubsection{Тема:} Введение. Определение чисел.
\subsubsection{Цель:} Познакомить ученика с понятием числа и его свойствами.

\subsection{Занятие 2}
\subsubsection{Тема:} Основные арифметические операции.
\subsubsection{Цель:} Обучить ученика складывать, вычитать, умножать и делить числа.

\subsection{Занятие 3}
\subsubsection{Тема:} Дроби.
\subsubsection{Цель:} Понять понятие дробей и освоить их сложение и вычитание.

\subsection{Занятие 4}
\subsubsection{Тема:} Десятичные дроби.
\subsubsection{Цель:} Изучить десятичные дроби и провести операции с ними.

\subsection{Занятие 5}
\subsubsection{Тема:} Действия с дробями и десятичными дробями.
\subsubsection{Цель:} Применить знания по дробям и десятичным дробям в решении задач.

\subsection{Занятие 6}
\subsubsection{Тема:} Геометрические фигуры.
\subsubsection{Цель:} Изучить основные геометрические фигуры и их свойства.

\subsection{Занятие 7}
\subsubsection{Тема:} Периметр и площадь.
\subsubsection{Цель:} Научить ученика находить периметр и площадь простых фигур.

\subsection{Занятие 8}
\subsubsection{Тема:} Уравнения.
\subsubsection{Цель:} Познакомить ученика с понятием уравнения и научить решать простейшие уравнения.

\subsection{Занятие 9}
\subsubsection{Тема:} Графики.
\subsubsection{Цель:} Понять назначение и построение графиков функций.

\subsection{Занятие 10}
\subsubsection{Тема:} Повторение и закрепление пройденного материала.
\subsubsection{Цель:} Применить знания по математике для решения разнообразных задач и задач на логику.

\end{document}
```
\end{document}