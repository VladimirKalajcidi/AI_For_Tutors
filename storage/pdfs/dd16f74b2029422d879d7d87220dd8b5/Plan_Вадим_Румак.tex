\documentclass{article}
\usepackage[utf8]{inputenc}
\usepackage[russian]{babel}
\usepackage{geometry}
\geometry{a4paper, margin=25mm}
\begin{document}

\documentclass{article}

\begin{document}

\section{Учебный план по математике для ученика Вадим}

\subsection{Характеристика ученика}
Вадим - очень ленивый ученик, который проводит большое количество времени в социальных сетях, особенно во <<Вконтакте>>.

\subsection{Цели учебного плана}
\begin{itemize}
    \item Побудить ученика проявлять интерес к математике.
    \item Помочь Вадиму развить математические навыки и умения.
    \item Поддерживать постоянный контакт с учеником для мотивации.
\end{itemize}

\subsection{План занятий}

\subsubsection{Занятие 1: Введение в математику}
\begin{itemize}
    \item Тема: Основные понятия и принципы математики.
    \item Цель: Заинтересовать ученика начать изучение математики.
    \item Задачи:
    \begin{itemize}
        \item Обсудить важность математики в повседневной жизни.
        \item Познакомить с базовыми понятиями.
    \end{itemize}
\end{itemize}

\subsubsection{Занятие 2: Арифметика}
\begin{itemize}
    \item Тема: Основы арифметики.
    \item Цель: Понять основные операции и их применение.
    \item Задачи:
    \begin{itemize}
        \item Разобрать основные арифметические действия.
        \item Решать практические задачи на сложение, вычитание, умножение и деление.
    \end{itemize}
\end{itemize}

\subsubsection{Занятие 3: Геометрия}
\begin{itemize}
    \item Тема: Основы геометрии.
    \item Цель: Понять геометрические фигуры и их свойства.
    \item Задачи:
    \begin{itemize}
        \item Изучить основные фигуры и их характеристики.
        \item Решать практические задачи на вычисление площадей и периметров.
    \end{itemize}
\end{itemize}

\subsubsection{Занятие 4: Уравнения и неравенства}
\begin{itemize}
    \item Тема: Решение уравнений и неравенств.
    \item Цель: Научиться решать простейшие уравнения и неравенства.
    \item Задачи:
    \begin{itemize}
        \item Разобрать методы решения уравнений и неравенств.
        \item Решать практические задачи на нахождение неизвестных.
    \end{itemize}
\end{itemize}

\subsubsection{Занятие 5: Статистика и вероятность}
\begin{itemize}
    \item Тема: Основы статистики и вероятности.
    \item Цель: Понять основные понятия статистики и вероятности.
    \item Задачи:
    \begin{itemize}
        \item Изучить понятия статистических данных и вероятности.
        \item Решать практические задачи на вероятность событий.
    \end{itemize}
\end{itemize}

\subsection{Дополнительные рекомендации}
\begin{itemize}
    \item Поддерживать постоянный контакт с учеником, мотивировать его на изучение математики.
    \item Использовать интересные и понятные примеры из повседневной жизни.
    \item Поощрять ученика за достижения и успехи в изучении математики.
\end{itemize}

\end{document}
\end{document}