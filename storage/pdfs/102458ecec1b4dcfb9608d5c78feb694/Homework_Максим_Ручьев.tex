\documentclass{article}
\usepackage[utf8]{inputenc}
\usepackage[russian]{babel}
\usepackage{geometry}
\geometry{a4paper, margin=25mm}
\begin{document}

Конечно, вот задание по следующей теме из учебного плана по предмету Математика для ученика Максима на уровне "Учиться в сунц нгу":

\documentclass{article}
\usepackage{amsmath}

\begin{document}

\title{Задание по математике}
\author{Ученик: Максим}
\date{}
\maketitle

\section*{Следующая тема: "Геометрические фигуры"}

\subsection*{Задача 1}
Найдите площадь круга с радиусом 10 единиц.

\subsection*{Задача 2}
Рассмотрим треугольник с углами 45°, 45°, 90°. Найдите длину гипотенузы, если катет равен 5 единицам.

\subsection*{Задача 3}
Дан равнобедренный треугольник с основанием 6 единиц и высотой 4 единицы. Найдите площадь этого треугольника.

\subsection*{Задача 4}
Рассмотрим прямоугольный параллелепипед с длиной 8 единиц, шириной 5 единиц и высотой 3 единицы. Найдите его объем.

\subsection*{Задача 5}
Известно, что периметр равностороннего треугольника равен 24 единицам. Найдите длину стороны треугольника.

\end{document}

Эти задачи помогут Максиму лучше понять геометрические фигуры и их свойства. Уверен, что ученик справится с ними на отлично!
\end{document}