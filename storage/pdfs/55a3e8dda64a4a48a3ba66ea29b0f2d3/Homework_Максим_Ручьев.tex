\documentclass{article}
\usepackage[utf8]{inputenc}
\usepackage[russian]{babel}
\usepackage{geometry}
\geometry{a4paper, margin=25mm}
\begin{document}

Конечно, вот домашнее задание по математике для ученика Максима, учитывающее замечания:

\[
\text{Домашнее задание по математике для ученика Максима}
\]

\[
\text{Уровень ученика: Учиться в сунц нгу}
\]

\textbf{Часть 1: Основные арифметические операции}

1. Решите следующие уравнения:
   a) \(5 + 3 \times 2\)
   b) \(12 - (4 \div 2)\)
   c) \((6 - 3) \times 4\)

2. Вычислите:
   a) \(3^2 + 4 \times 2\)
   b) \((8 - 3) \div 5\)
   c) \(\sqrt{16} - 2 \times 3\)

3. Переведите в десятичную дробь:
   a) \( \frac{3}{5} \)
   b) \( \frac{7}{8} \)
   c) \( \frac{2}{3} \)

\textbf{Часть 2: Геометрия}

4. Найдите периметр и площадь прямоугольника со сторонами 5 и 8 единиц.

5. Найдите площадь треугольника по формуле \(S = \frac{1}{2} \times a \times h\), где a - основание треугольника, h - высота.

6. Нарисуйте на координатной плоскости прямоугольник со сторонами, заданными точками \((0,0)\), \((5,0)\), \((5,3)\), \((0,3)\).

\textbf{Часть 3: Уравнения и неравенства}

7. Решите уравнения:
   a) \(2x + 5 = 11\)
   b) \(3(x - 4) = 15\)

8. Решите неравенства:
   a) \(2x + 7 < 15\)
   b) \(-3(x - 2) \geq 6\)

\textbf{Часть 4: Сложные задачи}

9. Поезд движется со скоростью 60 км/ч. Сколько времени ему потребуется, чтобы проехать 300 км?

10. В магазине продавались яблоки по цене 25 рублей за килограмм. Сколько стоит 2.5 кг яблок?

\textbf{Часть 5: Логика и анализ}

11. Если Анна старше, чем Мария, а Мария старше, чем Иван, то кто из них самый молодой?

12. Докажите или опровергните утверждение: "Если число делится на 3, то оно обязательно делится на 9".

\textbf{Дополнительные задачи}

13. Найдите площадь круга с радиусом 7 единиц.

14. Решите систему уравнений:
\[
\begin{cases}
2x + y = 5 \\
x - 3y = -2
\end{cases}
\]

15. Найдите значение выражения \(\frac{2}{3} - \frac{1}{4} + \frac{5}{6}\).

\[
\text{Желаю удачи в выполнении заданий!}
\]
\end{document}