\documentclass{article}
\usepackage[utf8]{inputenc}
\usepackage[russian]{babel}
\usepackage{geometry}
\geometry{a4paper, margin=25mm}
\begin{document}

```latex
\documentclass{article}
\usepackage[utf8]{inputenc}

\begin{document}

\section{Учебный план по Физике для ученика Владимир}

\subsection{Характеристики ученика}
Ученик Владимир умственно отсталый, но обладает финансовыми возможностями.

\subsection{Цели обучения}
\begin{itemize}
    \item Повышение интереса Владимира к изучению физики.
    \item Развитие базовых физических знаний и умений.
    \item Адаптация материала к индивидуальным особенностям ученика.
\end{itemize}

\subsection{Учебный план}

\subsubsection{Занятие 1: Введение в физику}
\begin{itemize}
    \item Цель: Знакомство с основными понятиями физики.
    \item Темы:
    \begin{itemize}
        \item Что такое физика?
        \item Основные законы природы.
    \end{itemize}
\end{itemize}

\subsubsection{Занятие 2: Механика}
\begin{itemize}
    \item Цель: Освоение основ механики.
    \item Темы:
    \begin{itemize}
        \item Движение и его типы.
        \item Законы Ньютона.
        \item Работа и энергия.
    \end{itemize}
\end{itemize}

\subsubsection{Занятие 3: Термодинамика и теплопередача}
\begin{itemize}
    \item Цель: Понимание основных принципов термодинамики.
    \item Темы:
    \begin{itemize}
        \item Температура и теплота.
        \item Законы термодинамики.
        \item Теплопроводность и тепловое излучение.
    \end{itemize}
\end{itemize}

\subsubsection{Занятие 4: Электричество и магнетизм}
\begin{itemize}
    \item Цель: Познакомить с основами электричества и магнетизма.
    \item Темы:
    \begin{itemize}
        \item Электрические заряды и поля.
        \item Законы Кулона и Ома.
        \item Магнитное поле и электромагнитная индукция.
    \end{itemize}
\end{itemize}

\subsubsection{Занятие 5: Оптика}
\begin{itemize}
    \item Цель: Понимание основ оптики и света.
    \item Темы:
    \begin{itemize}
        \item Оптические явления.
        \item Законы геометрической оптики.
        \item Волны и их свойства.
    \end{itemize}
\end{itemize}

\end{document}
```
\end{document}