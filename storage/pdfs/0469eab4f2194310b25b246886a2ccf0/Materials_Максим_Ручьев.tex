\documentclass{article}
\usepackage[utf8]{inputenc}
\usepackage[russian]{babel}
\usepackage{geometry}
\geometry{a4paper, margin=25mm}
\begin{document}

Конечно, вот обучающие материалы по теме «None» для Максима:

```latex
\documentclass{article}
\usepackage{amsmath}

\title{Учебные материалы: Тема «None»}
\author{Преподаватель}
\date{}

\begin{document}

\maketitle

\section{Теория}

\textbf{Определение:} Понятие «None» означает отсутствие значения или неопределенность.

В математике, ключевым понятием является «None», которое обозначает отсутствие какого-либо значения. Например, если переменная не имеет определенного значения, ей можно присвоить значение «None».

\section{Примеры}

\textbf{Пример 1:}
Если переменная $x$ не определена, можно присвоить ей значение «None»:
\[ x = \text{None} \]

\textbf{Пример 2:}
Функция, которая не возвращает явное значение, может вернуть «None»:
\begin{verbatim}
def my_function():
    print("Эта функция не возвращает значения")

result = my_function()
print(result)  # Вывод: None
\end{verbatim}

\section{Упражнения}

\textbf{Упражнение 1:}
Проверьте, что произойдет при выполнении следующего кода:
\begin{verbatim}
x = None
if x is None:
    print("Переменная x не определена")
else:
    print("Переменная x определена")
\end{verbatim}

\textbf{Упражнение 2:}
Напишите функцию, которая принимает число в качестве аргумента и возвращает его квадрат, если число нечетное, и «None», если число четное.

\end{document}
```

Эти материалы помогут Максиму понять понятие «None» в математике и научат его использовать в примерах и упражнениях.
\end{document}