\documentclass{article}
\usepackage[utf8]{inputenc}
\usepackage[russian]{babel}
\usepackage{geometry}
\geometry{a4paper, margin=25mm}
\begin{document}

\documentclass{article}
\usepackage{amsmath}

\begin{document}

\title{Домашнее задание по Физике}
\author{Ученик: Владимир}

\maketitle

\section*{Уровень ученика:}
Он умственно отсталый, но богатый.

\section*{Задачи:}

\begin{enumerate}
    \item Посчитайте скорость падения камня с высоты 10 метров. (Ускорение свободного падения примите равным $9.8 \, \text{м/c}^2$)
    
    \item Вагон длиной 20 м движется со скоростью 10 м/c. На какой расстояние вдоль пути движется конец вагона за время движения? (Считать, что вагон движется без ускорения)
\end{enumerate}

\end{document}
\end{document}