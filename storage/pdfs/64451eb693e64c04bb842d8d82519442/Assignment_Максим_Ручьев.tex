\documentclass{article}
\usepackage[utf8]{inputenc}
\usepackage[russian]{babel}
\usepackage{geometry}
\geometry{a4paper, margin=25mm}
\begin{document}

Конечно, вот задание по следующей теме из учебного плана по математике для ученика Максима:

\[
\text{Задание по математике для ученика Максима}
\]

\[
\text{Уровень ученика: Учиться в сунц нгу}
\]

\textbf{Тема: Системы уравнений}

1. Решите систему уравнений:

\[
\begin{cases}
2x - y = 3 \\
x + 3y = 5
\end{cases}
\]

2. Добавьте еще одну систему уравнений и решите ее:

\[
\begin{cases}
4x + 2y = 10 \\
3x - 2y = 7
\end{cases}
\]

\textbf{Нестандартные задачи:}

3. В городе N живет 150 тысяч человек. Если каждый второй житель города любит футбол, а каждый третий - баскетбол, то сколько человек любят оба эти спорта?

4. У Максима есть коробка со спичками. Если он возьмет 6 спичек, то в коробке останется 3 спички. Сколько спичек изначально было в коробке?

5. При каком значении параметра \(k\) система уравнений:

\[
\begin{cases}
2x + 3y = 7 \\
4x + ky = 13
\end{cases}
\]

будет иметь бесконечное множество решений?

6. Рассмотрим последовательность чисел: 2, 5, 10, 17, 26, ... Найдите следующее число в этой последовательности и объясните свое решение.

\textbf{Дополнительное задание:}

7. Решите задачу:

В одном отеле 40 номеров. Если каждый номер занимается 2 гостями, то на сколько больше гостей может быть заселено в отель, если каждый номер будет заниматься на 1 гостя меньше?

\[
\text{Желаю успехов в решении заданий!}
\]
\end{document}