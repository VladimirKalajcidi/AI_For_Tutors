\documentclass{article}
\usepackage[utf8]{inputenc}
\usepackage[russian]{babel}
\usepackage{geometry}
\geometry{a4paper, margin=25mm}
\begin{document}

```latex
\documentclass[a4paper,12pt]{article}
\usepackage[utf8]{inputenc}
\usepackage[russian]{babel}
\usepackage{amsmath}
\usepackage{amssymb}
\usepackage{enumitem}
\usepackage{geometry}
\geometry{left=2cm,right=2cm,top=2cm,bottom=2cm}

\begin{document}

\begin{center}
    {\LARGE \textbf{Термодинамика}}\\[1em]
    {\large Учебный материал для Максима}\\
    {\normalsize Уровень: Хороший парень, старательный}\\[1em]
    \hrule
\end{center}

\section*{Урок 1: Введение в термодинамику}

\subsection*{Цели урока}
\begin{itemize}[noitemsep]
    \item Познакомиться с основными понятиями термодинамики.
    \item Понять, что изучает термодинамика.
    \item Ознакомиться с основными параметрами состояния газов.
\end{itemize}

\subsection*{Теория}

Термодинамика — это раздел физики, изучающий процессы передачи и преобразования энергии, связанные с теплом и работой, а также свойства и поведение макроскопических систем (например, газов, жидкостей и твёрдых тел).

\textbf{Основные понятия:}

\begin{itemize}[noitemsep]
    \item \textbf{Термодинамическая система} — часть пространства, выбранная для изучения, отделённая от окружающей среды.
    \item \textbf{Термодинамическое состояние} — совокупность параметров, характеризующих систему (давление $p$, объем $V$, температура $T$, количество вещества).
    \item \textbf{Изопроцесс} — процесс, при котором один из параметров остаётся постоянным:
    \begin{itemize}[noitemsep]
        \item Изотермический: $T = \text{const}$
        \item Изобарический: $p = \text{const}$
        \item Изохорический: $V = \text{const}$
        \item Адиабатический: без теплообмена с окружающей средой ($Q=0$)
    \end{itemize}
\end{itemize}

\bigskip

\textbf{Идеальный газ} — упрощённая модель газа, в которой частицы не взаимодействуют друг с другом, кроме упругих столкновений.

Уравнение состояния идеального газа:
\[
pV = nRT,
\]
где
\begin{itemize}[noitemsep]
    \item $p$ — давление,
    \item $V$ — объем,
    \item $n$ — количество вещества (в молях),
    \item $R = 8{,}31\, \frac{\text{Дж}}{\text{моль} \cdot \text{К}}$ — универсальная газовая постоянная,
    \item $T$ — абсолютная температура (в Кельвинах).
\end{itemize}

\subsection*{Пример}

\textbf{Пример 1.} Газ занимает объём 2 л при давлении 100 кПа и температуре 300 К. Какое количество вещества $n$ содержится в газе?

\textit{Решение:}

Переведём объём в кубические метры: $V = 2\, \text{л} = 2 \times 10^{-3}\, \text{м}^3$.

Используем уравнение состояния идеального газа:
\[
n = \frac{pV}{RT} = \frac{100\,000 \times 2 \times 10^{-3}}{8{,}31 \times 300} \approx \frac{200}{2493} \approx 0{,}08\, \text{моль}.
\]

Ответ: $n \approx 0{,}08$ моль.

\subsection*{Упражнения}

\begin{enumerate}[label=\textbf{\arabic*.}, leftmargin=*, noitemsep]
    \item Идеальный газ занимает объём 5 л при давлении 50 кПа и температуре 27 °С. Найдите количество вещества газа.
    \item При изотермическом процессе объём газа увеличился в 3 раза. Как изменилась его давление?
    \item Опишите, что происходит с давлением, объёмом и температурой газа при изохорическом процессе.
    \item Объясните, что означает адиабатический процесс и приведите пример из жизни.
\end{enumerate}

\vspace{1em}

\textit{Подсказка:} Для задач, связанных с изменениями состояния газа, используйте уравнение состояния и законы сохранения энергии.

\bigskip

\textbf{Желаю успехов, Максим!}

\end{document}
```
\end{document}