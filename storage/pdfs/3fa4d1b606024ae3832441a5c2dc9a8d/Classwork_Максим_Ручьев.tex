\documentclass{article}
\usepackage[utf8]{inputenc}
\usepackage[russian]{babel}
\usepackage{geometry}
\geometry{a4paper, margin=25mm}
\begin{document}

Конечно, вот контрольная работа по математике для ученика Максима:

\[
\text{Контрольная работа по математике}
\]

\[
\text{Уровень ученика: Средний (учится в средней школе)}
\]

\[
\text{Время выполнения: 60 минут}
\]

\[
\text{Часть 1: Основные арифметические операции}
\]

\begin{enumerate}
    \item Вычислите:
    \begin{enumerate}
        \item \(7 + 4 \times 3 - 5\)
        \item \(18 - (5 \div 2) + 3\)
        \item \((8 - 2) \times 6 + 4\)
    \end{enumerate}
    
    \item Вычислите:
    \begin{enumerate}
        \item \(2^3 + 5 \times 2 - 4\)
        \item \((10 - 4) \div 3 + 1\)
        \item \(\sqrt{25} + 3 \times 2 - 7\)
    \end{enumerate}
    
    \item Переведите в десятичную дробь:
    \begin{enumerate}
        \item \( \frac{4}{7} \)
        \item \( \frac{5}{6} \)
        \item \( \frac{3}{4} \)
    \end{enumerate}
\end{enumerate}

\[
\text{Часть 2: Геометрия}
\]

\begin{enumerate}
    \item Найдите периметр и площадь прямоугольника со сторонами 6 и 9 единиц.
    \item Найдите площадь треугольника по формуле \(S = \frac{1}{2} \times a \times h\), где a - основание треугольника, h - высота. Для треугольника со сторонами 5, 12, 13.
\end{enumerate}

\[
\text{Часть 3: Уравнения и неравенства}
\]

\begin{enumerate}
    \item Решите уравнения:
    \begin{enumerate}
        \item \(3x + 7 = 22\)
        \item \(4(x - 3) = 20\)
    \end{enumerate}
    
    \item Решите неравенства:
    \begin{enumerate}
        \item \(3x + 5 < 17\)
        \item \(-2(x - 4) \geq 8\)
    \end{enumerate}
\end{enumerate}

\[
\text{Часть 4: Сложные задачи}
\]

\begin{enumerate}
    \item Поезд движется со скоростью 80 км/ч. Сколько времени ему потребуется, чтобы проехать 400 км?
    \item В магазине продавались апельсины по цене 30 рублей за килограмм. Сколько стоит 3.5 кг апельсинов?
\end{enumerate}

\[
\text{Часть 5: Логика и анализ}
\]

\begin{enumerate}
    \item Если Алексей старше, чем Никита, а Никита старше, чем Владимир, то кто из них самый молодой?
\end{enumerate}

Ученику Максиму желательно решить как можно больше заданий за отведенное время. Успехов в выполнении контрольной работы!
\end{document}