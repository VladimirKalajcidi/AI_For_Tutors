\documentclass{article}
\usepackage[utf8]{inputenc}
\usepackage[russian]{babel}
\usepackage{geometry}
\geometry{a4paper, margin=25mm}
\begin{document}

Данное домашнее задание разнообразно и предлагает ученику Максиму интересные задачи для развития его математических навыков. Уровень сложности соответствует его текущим знаниям и поможет ему продолжать изучение математики.

\documentclass{article}
\usepackage{amsmath}

\begin{document}

\title{Домашнее задание по математике}
\author{Ученик: Максим}
\date{}
\maketitle

\section*{Задание 1: Оригами и математика}

При изготовлении оригами используются различные геометрические фигуры. Ваша задача: построить оригами из квадрата бумаги размером 15x15 см, разрезав его по диагонали. Определите площадь получившихся фигур и сравните их.

\section*{Задание 2: Математика в кулинарии}

Вы готовите торт по рецепту, где указаны доли ингредиентов в процентах. Например, для крема нужно 1/4 часть сахара и 3/4 часть масла от общего веса. Если общий вес крема 500 г, сколько грамм сахара и масла нужно добавить?

\section*{Задание 3: Геометрические фигуры вокруг нас}

Изучите правила построения 3D-моделей простых геометрических фигур (куб, пирамида, цилиндр) из бумаги или картона. Попробуйте создать одну из них и опишите процесс.

\section*{Задание 4: Криптография и математика}

Изучите принцип работы шифра Цезаря, где каждая буква заменяется на букву с определенным сдвигом в алфавите. Зашифруйте и расшифруйте свое имя, используя шифр Цезаря с определенным ключом.

\section*{Задание 5: Математика в музыке}

Изучите основы музыкальной гармонии и теории звука. Попробуйте создать музыкальную мелодию, используя математические принципы (например, соотношение частот звуков).

\end{document}
\end{document}