\documentclass{article}
\usepackage[utf8]{inputenc}
\usepackage[russian]{babel}
\usepackage{geometry}
\geometry{a4paper, margin=25mm}
\begin{document}

Конечно, вот домашнее задание по математике для ученика Вадима:

\[
\begin{align*}
&\text{1. Решите уравнение: } 5x + 3 = 18 \\
&\text{2. Посчитайте } 3 \times \frac{2}{5} \\
&\text{3. Найдите среднее арифметическое чисел 8, 12, 15, 21} \\
&\text{4. Решите задачу: Если 20\% числа равно 30, то какое это число?} \\
&\text{5. Сложите дроби: } \frac{3}{4} + \frac{2}{3} \\
&\text{6. Найдите площадь прямоугольника со сторонами 5 см и 8 см} \\
&\text{7. Решите задачу: Площадь треугольника равна 24 кв. см, а его высота равна 6 см. Найдите основание треугольника} \\
&\text{8. Найдите длину окружности, если её радиус равен 7 см} \\
&\text{9. Сколько градусов в 1/4 оборота?} \\
&\text{10. Найдите объем параллелепипеда, если его длина 4 см, ширина 6 см, высота 10 см} \\
&\text{11. Разложите на множители: } x^2 + 5x + 6 \\
&\text{12. Решите систему уравнений: } \begin{cases} 2x + y = 5 \\ x - y = 1 \end{cases} \\
&\text{13. Решите уравнение: } 2^{x-1} = 8 \\
&\text{14. Найдите значение выражения при } x = 2: 3x^2 - 4x + 5 \\
&\text{15. Постройте график функции } y = 2x + 3 \\
&\text{16. Решите квадратное уравнение: } x^2 - 7x + 12 = 0 \\
&\text{17. Найдите сумму бесконечно убывающей геометрической прогрессии с первым членом 8 и знаменателем }\frac{1}{2} \\
&\text{18. Решите задачу: В треугольнике ABC угол A равен 50 градусов, а сторона BC равна 12 см. Найдите сторону AC} \\
&\text{19. Найдите значение выражения } \frac{5!}{3!} \\
&\text{20. Решите уравнение: } \sqrt{3x + 1} = 5 \\
&\text{21. Какая из дробей больше: } \frac{3}{5} \text{ или } \frac{4}{7}? \\
&\text{22. Найдите ошибку в решении уравнения: } 2x - 5 = 11 \Rightarrow x = \frac{11}{2} + 5 \\
&\text{23. Решите задачу: Если 1 яблоко стоит 10 рублей, то сколько стоит 5 яблок?} \\
&\text{24. Найдите сумму ряда: } 1 + \frac{1}{2} + \frac{1}{4} + \frac{1}{8} + \ldots \\
&\text{25. Если } a = 3 \text{ и } b = 2, \text{ то что больше: } a^b \text{ или } b^a? \\
&\text{26. Какие свойства имеют параллелограммы?} \\
&\text{27. Чем отличаются правильный и произвольный многоугольники?} \\
&\text{28. Какие способы решения уравнений существуют?} \\
&\text{29. В чем отличие линейной и квадратичной функций?} \\
&\text{30. Какое значение имеет число } \pi \text{ в математике?} \\
\end{align*}
\]

Желаю успешно выполнить все задания!
\end{document}