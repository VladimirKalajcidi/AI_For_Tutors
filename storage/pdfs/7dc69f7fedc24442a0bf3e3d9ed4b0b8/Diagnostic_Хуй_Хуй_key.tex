\documentclass{article}
\usepackage[utf8]{inputenc}
\usepackage[russian]{babel}
\usepackage{geometry}
\geometry{a4paper, margin=25mm}
\begin{document}

Для составления полного LaTeX-документа с ключом ответов к диагностическому тесту по математике для ученика, воспользуемся следующим кодом:

```latex
\documentclass{article}
\usepackage{amsmath}

\begin{document}

\section*{Ключ ответов к диагностическому тесту по математике}

\subsection*{Часть 1: Базовые понятия}
1. Наибольшее общее кратное (НОК) двух чисел - это наименьшее число, которое делится на оба этих числа без остатка.
2. \((-3) \times 7 = -21\).
3. Простыми числами являются 23 и 31.
4. Десятичная дробь - это дробное число, где дробная часть записана после десятичной запятой.
5. \(x = 6\).

\subsection*{Часть 2: Алгебра}
6. \(x = 2, y = 1\).
7. \(x^2 - 4 = (x - 2)(x + 2)\).
8. Корнями уравнения являются 2 и 3.
9. При \(x = 3\), значение выражения равно 12.
10. \(x \leq -2\).

\subsection*{Часть 3: Геометрия}
11. Прямоугольник имеет стороны, образующие прямые углы, квадрат - это прямоугольник, у которого все стороны равны.
12. Периметр треугольника равен 15.
13. Сумма углов треугольника равна 180 градусов.
14. Площадь круга равна \(\pi r^2\), где \(r\) - радиус круга.
15. Синус угла в прямоугольном треугольнике - это отношение длины противоположенного катета к гипотенузе.

\subsection*{Часть 4: Смешанные задачи}
16. Уравнение \(x^2 - 6x + 9 = 0\) имеет корень \(x = 3\), что соответствует вершине параболы, аналогичной квадрату с вершиной в точке (3, 0).
17. Площадь квадрата увеличится в 9 раз.
18. Выражение \(a^2 - b^2\) можно преобразовать к виду \((a - b)(a + b)\).
19. Среднее арифметическое чисел 4, 7 и 10 равно 7.
20. Формула для нахождения объема параллелепипеда: \(V = l \times w \times h\), где \(l\), \(w\), \(h\) - длина, ширина и высота соответственно.

\end{document}
```

Этот код создаст полный LaTeX-документ с ключом ответов ко всем вопросам диагностического теста по математике для ученика.
\end{document}