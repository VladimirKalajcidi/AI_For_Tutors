\documentclass{article}
\usepackage[utf8]{inputenc}
\usepackage[russian]{babel}
\usepackage{geometry}
\geometry{a4paper, margin=25mm}
\begin{document}

Конечно, вот задание по следующей теме из учебного плана по математике для ученика Максима на уровне "Учиться в сунц нгу":

\documentclass{article}
\usepackage{amsmath}

\begin{document}

\title{Задание по математике для ученика Максима}
\author{}
\date{}
\maketitle

\section*{Тема: Сложные задачи}

\subsection*{Задача 1}
У Максима есть садовый участок, площадь которого равна 120 квадратных метров. Он хочет засадить на этом участке овощной город. Если площадь, занятая одним грядкой, составляет 8 квадратных метров, сколько грядок с овощами сможет разместить Максим на своем участке?

\subsection*{Задача 2}
Максим решил сходить в магазин за продуктами. Он купил 2 кг муки по цене 40 рублей за килограмм, 1.5 кг сахара по цене 30 рублей за килограмм и 0.5 кг масла по цене 80 рублей за килограмм. Сколько денег он потратил на продукты?

\subsection*{Задача 3}
На футбольном турнире участвовало 8 команд. Каждая команда сыграла с каждой по одному матчу. Сколько всего матчей было сыграно на турнире?

\subsection*{Задача 4}
Максим купил автомобиль за 500 000 рублей. Каждый год стоимость его автомобиля уменьшается на 10%. Через сколько лет автомобиль потеряет половину своей первоначальной стоимости?

\subsection*{Задача 5}
У Максима есть коллекция марок. Если у него 56 марок, и каждый месяц он покупает по 3 новых марки, то через сколько месяцев у него будет в два раза больше марок, чем сейчас?

\end{document}
\end{document}