\documentclass{article}
\usepackage[utf8]{inputenc}
\usepackage[russian]{babel}
\usepackage{geometry}
\geometry{a4paper, margin=25mm}
\begin{document}

\documentclass{article}
\usepackage[utf8]{inputenc}

\begin{document}

\section*{Учебный план по Математике для ученика Вадим}

\subsection*{Неделя 1: Введение в алгебру}
\begin{itemize}
    \item Цель: Познакомить ученика с основными понятиями алгебры.
    \item Темы:
    \begin{itemize}
        \item Введение в алгебраические выражения.
        \item Операции с алгебраическими выражениями.
    \end{itemize}
\end{itemize}

\subsection*{Неделя 2: Решение уравнений}
\begin{itemize}
    \item Цель: Научить ученика решать уравнения.
    \item Темы:
    \begin{itemize}
        \item Одношаговые уравнения.
        \item Многошаговые уравнения.
    \end{itemize}
\end{itemize}

\subsection*{Неделя 3: Геометрия}
\begin{itemize}
    \item Цель: Ввести ученика в мир геометрии.
    \item Темы:
    \begin{itemize}
        \item Основные понятия геометрии.
        \item Решение задач на построение.
    \end{itemize}
\end{itemize}

\subsection*{Неделя 4: Системы уравнений}
\begin{itemize}
    \item Цель: Обучить ученика работе с системами уравнений.
    \item Темы:
    \begin{itemize}
        \item Методы решения систем уравнений.
        \item Практические задачи на системы уравнений.
    \end{itemize}
\end{itemize}

\subsection*{Неделя 5: Введение в теорию вероятностей}
\begin{itemize}
    \item Цель: Познакомить ученика с основами теории вероятностей.
    \item Темы:
    \begin{itemize}
        \item Основные понятия теории вероятностей.
        \item Решение задач на вероятности.
    \end{itemize}
\end{itemize}

\end{document}
\end{document}