\documentclass{article}
\usepackage[utf8]{inputenc}
\usepackage[russian]{babel}
\usepackage{geometry}
\geometry{a4paper, margin=25mm}
\begin{document}

\documentclass{article}
\usepackage{amsmath}

\begin{document}

\title{Домашнее задание по математике}
\author{Ученик: Максим}
\date{}
\maketitle

\section*{Тема: Следующая тема из учебного плана}

\subsection*{Задание 1}
Выразите число 37 в виде суммы двух квадратов.

\subsection*{Задание 2}
Найдите сумму первых 10 членов арифметической прогрессии, если первый член равен 3, а разность равна 4.

\subsection*{Задание 3}
Докажите, что корень квадратный из числа 2 — иррациональное число.

\subsection*{Задание 4}
Решите уравнение:
\begin{equation*}
    2^x - 3 \cdot 2^{-x} = 3
\end{equation*}

\subsection*{Задание 5}
Найдите сумму бесконечного ряда:
\begin{equation*}
    \sum_{n=1}^{\infty} \frac{1}{2^n}
\end{equation*}

\end{document}
\end{document}