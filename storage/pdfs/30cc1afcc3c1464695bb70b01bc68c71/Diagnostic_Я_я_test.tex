\documentclass{article}
\usepackage[utf8]{inputenc}
\usepackage[russian]{babel}
\usepackage{geometry}
\geometry{a4paper, margin=25mm}
\begin{document}

Хорошо, вот мой предложенный масштабный диагностический тест по математике:

### Часть 1: Основные понятия

1. Что такое действительные числа?
2. Что такое дробь? Приведите примеры действий с дробями.
3. Как определить четность или нечетность числа?
4. Что такое простые и составные числа? Приведите примеры.
5. Какие свойства имеют операции сложения, вычитания, умножения и деления?

### Часть 2: Алгебраические выражения

6. Что такое алгебраическое выражение? Приведите примеры.
7. Как упрощать алгебраические выражения?
8. Решите уравнение: \(3x + 5 = 17\).
9. Чем отличаются уравнения от неравенств?
10. Какой метод использовать для решения системы уравнений?

### Часть 3: Геометрия

11. Какие виды углов существуют? Приведите примеры.
12. Что такое параллельные и перпендикулярные прямые?
13. Как найти площадь треугольника? Приведите формулу.
14. Чем отличаются прямоугольник, квадрат и ромб?
15. Как найти объем и площадь поверхности цилиндра?

### Часть 4: Сложные задачи

16. Решите задачу: Петя купил яблоки по 15 рублей за килограмм и апельсины по 25 рублей за килограмм. Если у него всего 2 кг фруктов и он заплатил 60 рублей, сколько кг яблок и сколько кг апельсинов он купил?
17. Известно, что \(a + b = 9\) и \(ab = 14\). Найдите значения переменных \(a\) и \(b\).
18. Решите задачу: Поезд движется со скоростью 60 км/ч. Через сколько часов он преодолеет расстояние в 240 км?
19. Найдите сумму первых 10 членов арифметической прогрессии с первым членом 4 и разностью 3.
20. Какие признаки простоты числа вы знаете и как их применять?

### Часть 5: Доказательства и Обоснования

21. Докажите, что сумма углов треугольника равна 180 градусов.
22. Обоснуйте, почему умножение на нуль дает ноль.
23. Докажите, что корень квадратный из числа неотрицательный.
24. Обоснуйте, как решать уравнения с помощью метода подстановки.
25. Докажите, что произведение двух четных чисел всегда четное.

### Часть 6: Открытые вопросы

26. Какие темы в математике вам кажутся наиболее сложными и почему?
27. Какие методы и приемы решения задач вам больше всего нравятся?
28. Какие области математики вам интересны для изучения в будущем?

Этот тест позволит оценить широту и глубину знаний ученика по математике, а также выявит его уровень подготовки и способности к применению знаний на практике.
\end{document}