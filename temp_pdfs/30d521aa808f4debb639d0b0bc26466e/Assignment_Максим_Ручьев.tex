\documentclass{article}
\usepackage[utf8]{inputenc}
\usepackage[russian]{babel}
\usepackage{geometry}
\geometry{a4paper, margin=25mm}
\begin{document}

Для ученика Максима, уровень: "Учиться в сунц нгу", подготовлено домашнее задание по следующей теме из учебного плана по предмету Математика:

\documentclass{article}
\usepackage{amsmath}

\begin{document}

\title{Домашнее задание по Математике}
\author{Максим}
\date{}
\maketitle

\section*{Задания}

\begin{enumerate}
    \item Найдите решение системы уравнений:
    \begin{align*}
        3x - 2y &= 4 \\
        2x + y &= 5
    \end{align*}

    \item Вычислите значение выражения:
    \[
    \sqrt{25} + \frac{3}{2} \times (7 - 4)
    \]

    \item Найдите площадь круга с радиусом 10 единиц.

    \item Решите задачу: В саду растут яблони, груши и сливы. Если на яблонях 30 яблок, на грушах в 2 раза меньше, чем на яблонях, а на сливах в 3 раза больше, чем на грушах, сколько всего фруктов в саду?

    \item Подбросьте монетку два раза. Какова вероятность выпадения хотя бы одного орла?

\end{enumerate}

\end{document}
\end{document}